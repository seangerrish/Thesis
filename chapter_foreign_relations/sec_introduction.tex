The written history of foreign relations is peppered with the bias of
politics and hindsight.  Because historians currently have limited
resources to review thousands of relevant news sources, research may
be biased by popular knowledge or even politics and culture. 

In this chapter we will use the text of newspaper articles to infer a
history of the relationships between different nations. An assumption
of our work is that the tension between two nations -- or a warm and
robust relationship between them -- is reflected by the language we
use to discuss them. Using this assumption, we outline a model that is
able to infer the relationship between pairs of countries whose
relationship has not been observed in training.

%, an idea
%inspired by relational topic models and multidimensional scaling
%\cite{chang:2009}.

An advantage of a text-based approach to history is that we can
incorporate information from all articles of a given collection with
modest computational cost.  This means that historians and political
scientists can then search and review thousands of historical
documents to identify forgotten or overlooked ``blips'' in history.

% We infer sentiment with the supervision of voting on UN resolutions.
% Learning sentiment based on voting patterns provides an objective
% measure of the political goals of different countries.

% Potential idea: can use filtering instead of smoothing to see
% how things look "as they happen".
