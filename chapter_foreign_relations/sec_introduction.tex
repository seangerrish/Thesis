In this chapter we will use the text of newspaper articles to infer a
history of the relationships between different nations.  Just as with
the previous chapter, a computational model of textwill allow us to
incorporate information from all articles of a given collection with
modest computational cost.  This means that historians and political
scientists can then search and review thousands of historical
documents to identify forgotten or overlooked ``blips'' in history.
Because historians currently have limited resources to review
thousands of relevant news sources, research may be colored by popular
knowledge or even politics and culture.

An assumption of our work is that the tension between two nations --
or a warm and robust relationship between them -- is reflected by the
language that is used to discuss them.  In developing this assumption,
we will discuss two models designed to infer the relationships between
pairs of countries.

%, an idea
%inspired by relational topic models and multidimensional scaling
%\cite{chang:2009}.

An important development in this chapter is that we will incorporate
explicit assumptions about individual countries within our collection.
Our assumptions---known in the literature as multidimensional
scaling---will allow us to make predictions about the relationships
between pairs of countries over time.  By linking this assumption to
text, we will also be able determine the relationship between
countries using the text of documents.  While the last chapter's
Document Influence Model allowed us to describe themes over time and
individual documents' influence on these themes, the assumptions we
will make in this chapter allow us to create a more rich story about
specific entities--countries--over time.

% We infer sentiment with the supervision of voting on UN resolutions.
% Learning sentiment based on voting patterns provides an objective
% measure of the political goals of different countries.

% Potential idea: can use filtering instead of smoothing to see
% how things look "as they happen".

\section{Organization of this chapter}

In the next two sections we will develop two computational models that
link the text of a news source to the relationships between countries.
We begin with a model which infers these relationships by using using
two sources of labels about the sentiment between pairs of countries.
We will develop a set of spatio-temporal assumptions that allow us to
describe the sentiment between countries by inspecting their relative
positions in a latent space.  We will demonstrate that modeling
countries with this model allows us to create a history of foreign
relations over time.  Importantly, we will show that the latent-space
assumption allows us to better represent the sentiment observed
between countries than a text-only model would allow.

After developing a supervised model, we invert this question and ask:
what sentiment is implied by the text alone of news articles?  We will
develop an unsupervised model of the relationship between countries to
to qualitatively describe these relationships.  We then demonstrate a
connection between the unsupervised relationships and the sentiment
labels we had used for the supervised model.
