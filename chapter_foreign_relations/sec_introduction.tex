In this chapter we use the text of newspaper articles to infer a
history of the relationships between different nations.  An assumption
of our work is that the tension between two nations---or a warm and
robust relationship between them---is reflected by the language that
is used to discuss them.  In developing this assumption, we
discuss two models designed to infer the relationships between pairs
of nations.

\subsection*{Text and latent spaces}
The basic unit of analysis in this chapter is paragraphs of text
from newspaper articles which discuss pairs of nations.  We choose
paragraphs because they are small enough to have just one or two
concrete ideas but large enough to describe interesting
relationships.

We use some of the same ideas presented in the last chapter to
model the text of these paragraphs, but we use one of the
primitives introduced in Chapter 2 to model relationships between
pairs of nations.  This allows us to build a history of nations'
relationships over time.  An advantage of a text-based approach to
history is that we can incorporate information from all articles of a
given collection with modest computational cost.  This means that
historians and political scientists can then search and review
thousands of historical documents at the push of a button---or
identify forgotten and overlooked incidents in history.

The primitive from Chapter 2 that we use amounts to an assumption
that each nation can be summarized by its position in a latent space,
so that the sentiment between two nations is determined (up to
stochasticity) by the relationship between their positions in this
latent space.  By making this assumption, we gain two benefits:
the ability to interpret these nations' positions, since they provide
statistically meaningful summaries of these nations' positions; and
the ability to make predictions about the relationships between
nations, based on their latent positions. While the last chapter's
Document Influence Model allowed us to discover themes which evolved
over time and individual documents' influence on these themes, the
assumptions we make in this chapter allow us to create a
more rich story about the interaction of specific textual
entities---nations---over time.

% We infer sentiment with the supervision of voting on UN resolutions.
% Learning sentiment based on voting patterns provides an objective
% measure of the political goals of different nations.

% Potential idea: can use filtering instead of smoothing to see
% how things look "as they happen".

\subsection*{Organization of this chapter}

In the next two sections we develop several computational models that
link the text of a news source to the relationships between nations.

We begin with a model which infers these relationships by using
two sources of labels about the the relationship, or sentiment,
between pairs of nations: expert labels and labels assigned by lay
paid ``workers''.  To design this model, we develop a set of
spatio-temporal assumptions that allow us to describe the sentiment
between nations by inspecting their relative positions in this
latent space (and, inversely, to interpret their positions based on
observed sentiment).  We demonstrate that modeling nations in
this way allows us to create a history of foreign relations over time.
Importantly, we demonstrate that the sentiment inferred from two
very different sources of sentiment labels leads to strikingly similar
measures of inter-state sentiment.

After developing this supervised model, we invert this question and
ask: what sentiment is implied by the text alone of news articles?  To
answer this question, we describe an unsupervised model of the
relationship between nations to qualitatively describe these
relationships.  We then demonstrate a connection between the
unsupervised relationships and the sentiment labels we had used for
the supervised model.
