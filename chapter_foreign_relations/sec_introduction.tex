% \section*{Introduction}

The written history of foreign relations is peppered with the bias of
politics and hindsight.  Because historians currently have limited
resources to review thousands of relevant news sources, research may
be biased by popular knowledge or even politics and culture. 

In this chapter we will use the text of newspaper articles to infer a
history of the relationships between different nations. An assumption
of our work is that the tension between two nations -- or a warm and
robust relationship between them -- is reflected by the language used
to discuss them. Using this assumption, we will outline a model that
is able to infer the relationship between pairs of countries whose
relationship has not been observed in training.

%, an idea
%inspired by relational topic models and multidimensional scaling
%\cite{chang:2009}.

\subsection*{Text and latent spaces}
The unit of analysis in this chapter will be paragraphs from news
articles---one of the the smallest units of discussion about a pair of
countries; and we will use some of the same ideas presented in the last chapter to model this text: mixed-membership models of each document.  An advantage of a text-based approach to history is that
we can incorporate information from all articles of a given collection
with modest computational cost.  This means that historians and
political scientists can then search and review thousands of
historical documents to identify forgotten or overlooked incidents in
history.

An important additional assumption that we will make is that each
nation can be summarized by its position in a latent space, so that
the sentiment between two nations is determined (up to stochasticity)
by the relationship between their positions in this latent space.  By
making this assumption, we will gain two benefits: the ability to
interpret these countries' positions, since they provide meaningful
summaries of these countries' positions; and the ability to make
predictions about the relationships between countries, based on their
latent positions.

We begin this chapter by outlining a model of foreign relations.  In
this model, we will use both expert and novice labels of the sentiment
between pairs of countries attached to each snippet of text.  Using
this model, we will demonstrate that embedding countries into a latent
space improves our ability to predict sentiment of heldout documents.
After analyzing this supervised model, we will turn the question
around and ask, ``in the absense of supervision, what type of
relationships best describe the interaction between countries?''  As
we will demonstrate, this relationship coincides with expert labels.

% We infer sentiment with the supervision of voting on UN resolutions.
% Learning sentiment based on voting patterns provides an objective
% measure of the political goals of different countries.

% Potential idea: can use filtering instead of smoothing to see
% how things look "as they happen".
