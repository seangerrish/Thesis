In this chapter we will use the text of newspaper articles to infer a
history of the relationships between different nations.  Just as with
the previous chapter, a computational model of text will allow us to
incorporate information from all articles of a given collection with
modest computational cost.  This means that historians and political
scientists can then search and review thousands of historical
documents to identify forgotten or overlooked ``blips'' in history.
Because historians currently have limited resources to review
thousands of relevant news sources, research may be colored by popular
knowledge or even politics and culture.

An assumption of our work is that the tension between two nations --
or a warm and robust relationship between them -- is reflected by the
language that is used to discuss them.  In developing this assumption,
we will discuss two models designed to infer the relationships between
pairs of countries.

\subsection*{Text and latent spaces}
The unit of analysis in this chapter will be paragraphs from news
articles---one of the the smallest units of discussion about a pair of
countries; and we will use some of the same ideas presented in the last chapter to model this text: mixed-membership models of each document.  An advantage of a text-based approach to history is that
we can incorporate information from all articles of a given collection
with modest computational cost.  This means that historians and
political scientists can then search and review thousands of
historical documents to identify forgotten or overlooked incidents in
history.

An important additional assumption that we will make is that each
nation can be summarized by its position in a latent space, so that
the sentiment between two nations is determined (up to stochasticity)
by the relationship between their positions in this latent space.  By
making this assumption, we will gain two benefits: the ability to
interpret these countries' positions, since they provide meaningful
summaries of these countries' positions; and the ability to make
predictions about the relationships between countries, based on their
latent positions. While the last chapter's Document Influence Model
allowed us to describe themes over time and individual documents'
influence on these themes, the assumptions we will make in this
chapter allow us to create a more rich story about specific
entities--countries--over time.

% We infer sentiment with the supervision of voting on UN resolutions.
% Learning sentiment based on voting patterns provides an objective
% measure of the political goals of different countries.

% Potential idea: can use filtering instead of smoothing to see
% how things look "as they happen".

\subsection{Organization of this chapter}

In the next two sections we will develop two computational models that
link the text of a news source to the relationships between countries.

We begin with a model which infers these relationships by using using
two sources of labels about the sentiment between pairs of countries.
We will develop a set of spatio-temporal assumptions that allow us to
describe the sentiment between countries by inspecting their relative
positions in a latent space.  We will demonstrate that modeling
countries with this model allows us to create a history of foreign
relations over time.  Importantly, we will show that the latent-space
assumption allows us to better represent the sentiment observed
between countries than a text-only model would allow.

After developing a supervised model, we invert this question and ask:
what sentiment is implied by the text alone of news articles?  To
answer this question, we will develop an unsupervised model of the
relationship between countries to to qualitatively describe these
relationships.  We then demonstrate a connection between the
unsupervised relationships and the sentiment labels we had used for
the supervised model.
