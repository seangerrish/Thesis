\section{A supervised model of dyadic sentiment}

\label{section:foreign_relations_supervised_model}

In the last chapter we described a model for identifying influential
documents.  A defining feature of that model was that it was
unsupervised; only after fitting the model could we compare the
inferred influence of an article with the number of citations it had
recieved.  In this section we will take a more direct approach,
fitting a model with labels \emph{defined} to represent the
information we are seeking -- sentiment.

At the same time, we will make further assumptions about the object of
discussion in the text. We will assume in particular that each country
can be described by a vector in some latent space.  The relationship
between two countries is then determined (up to stochasticity) by the
relationship between these countries' positions in this latent space
(models that make this assumption are sometimes referred to as latent
space models or spatial models).

A spatial a model provides us with two benefits. First, it provides
interpretability: we will define the model so that nations with
similar positions in this latent space tend to interact more
positively, while nations further apart tend to have more tension in
their relationship.  Second, it allows us to draw on existing work
from multidimensional scaling, which has been used successfully in
both political science \citep{martin:2002,jackman:2001} and social
network modeling \citep{hoff:2002,chang:2009}.

\subsection{Inferring sentiment from text}
\label{section:text_regression}
When a news source discusses the relationship between these nations,
the author's choice of words $\bm w_d \in \mathbb{N^+}^V$ reflects the relationship
between the countries.  We model this sentiment with the text of the
article $d$.  Using text regression \citep{kogan:2009}, we model
sentiment using the wordcounts $\bm w_d$ of the article:
\begin{align}
  s_d | \bm w_d, \bm \beta \sim \mathcal{N}( \bm w_d^T \bm \beta,
  \sigma_W^2 ) \\
  \bm \beta \sim \mathcal{N}(0, \sigma_\beta^2 ).
  \label{eq:sentiment_text}
\end{align}
For the remainder of this section, we will assume that $\bm \beta$ is
observed, and therefore that the distribution of $s_d$ is a Gaussian with mean
$\bm w_d^T \bm \beta$.  We describe how to fit $\bm \beta$ with
\myeq{sentiment_text} and human labels in
Section~\ref{section:mturk}.

\subsection{A temporal model of interaction}
We formalize the latent space assumption by letting each country $c$
take a position $\bm x_c \in \mathbb{R}^p$ in a space of latent
political sentiment. The relationship between two countries $c_1, c_2$
will be described by a scalar $s_{c_1,c_2} \in \mathbb{R}$.  This
sentiment is determined by the interaction of their positions:
$s_{c_1, c_2} = \mathcal{F}(\bm x_{c_1}, \bm x_{c_2})$, for some
suitable function $\mathcal{F}: \mathbb{R} \times \mathbb{R}
\rightarrow \mathbb{R}$. When $c_1$ and $c_2$ are similar (as measured
by $\mathcal{F}$), their sentiment $s_d$ will be positive; if they are
dissimilar, their sentiment will be negative.  More extreme values
indicate stronger sentiment.

\begin{figure}
  \center
  \vspace{-55pt}
  \includegraphics[width=0.5\textwidth]{chapter_foreign_relations/figures/countries_gm.pdf}
  \caption{A time-series model of countries' interactions.
    Pseudo-observations of ``zero'' are added for regularization.
    Amazon Mechanical Turk labels are used to fit $\bm \beta$, which is
    used to infer unobserved sentiments.}
  \label{fig:fa_gm}
\end{figure}

Foreign relations are not static; nations' alliances and preferences
change over time with the evolution of economies, technology, and
culture.  Therefore we make this a fully temporal model by
allowing each country's mean position $\bar x_{c,t}$ to drift over
time with the Markov transition
\begin{align}
  \bar x_{c,t} | \bar x_{c,t-1} \sim \mathcal{N}(\bar x_{c,t-1}, \sigma_{\mbox{\tiny chain}}^2),
\end{align}
as shown in Figure~\ref{fig:fa_gm}. At any time $t$, we may observe
the relationship between states $c_1$ and $c_2$ in an article $d$.  As
before, the distribution of the sentiment between these countries is
entirely specified by their positions at this time:
\begin{align}
  x_{c_1,d} \sim \mathcal{N}(\bar x_{c_1, t}, \sigma_D^2) \nonumber \\
  x_{c_2,d} \sim \mathcal{N}(\bar x_{c_2, t}, \sigma_D^2) \nonumber \\
  s_d := \mathcal{F}(\bm x_{c_1,d}, \bm x_{c_2,d}), \label{eq:sentiment_space}
\end{align}
where we interpret $s_d$ as the sentiment between $c_1$ and $c_2$ as
reflected by article $d$.

For the purposes of countries' positions in the latent-space model,
however, we will use the symmetry of the Gaussian to model text as if
it were conditioned on sentiment: $p(s_d | \bm w_d^T \bm \beta,
\sigma_W^2) = p( \bm w_d^T \bm \beta | s_d, \sigma_W^2 )$.  This
allows us to reconcile \myeq{sentiment_text} and
\myeq{sentiment_space}, so that the distribution of sentiment
conditioned on text and countries' positions is:
\begin{align}
  p(s_d | \bm w_d, \bm \beta, x_{c_1,d}, x_{c_2,d}) & \propto
  p(\bm w_d^T \bm \beta | s_d, \sigma_W^2 )
  p(x_{c_1,d} | \bar x_{c_1, t}, \sigma_D^2)
  p(x_{c_2,d} | \bar x_{c_2, t}, \sigma_D^2) \nonumber \\
  & \hspace{20pt} \mbox{ such that } s_d = \mathcal{F}(x_{c_1,t},
  x_{c_2,t}) \nonumber \\
  & = 
  p(\bm w_d^T \bm \beta |
    \mathcal{F}(x_{c_1,t}, x_{c_2,t}), \sigma_W^2 )
  p(x_{c_1,d} | \bar x_{c_1, t}, \sigma_D^2)
  p(x_{c_2,d} | \bar x_{c_2, t}, \sigma_D^2).
\end{align}

% In addition, a UN resolution may come up for vote at any time.  States
% cast a vote based on their current positions:
% \begin{align}
% x_{c_1,d} \sim N(\bar x_{c_1, t}, \sigma_D^2) \nonumber \\
% p(v_{cr}) = \sigma(x_{c_2, t} b_r + a_r) \nonumber \\
% \end{align}

% \begin{wrapfigure}{r}{0.4\textwidth}
%   \includegraphics[width=0.4\textwidth]{figs/countries_gm.pdf}
%   \caption{The full time-series model of interaction by countries.
%     The large plate shows replication of a Markov chain for each
%     country.  Certain countries interact at each epoch -- possibly
%     multiple times -- with sentiment $s$.}
%   \label{fig:countries_by_ip}
% \end{wrapfigure}

\paragraph{A brief comment on notation.} Before proceeding to
inference and the experimental validation of this model, we pause to
summarize our use of notation.  In this chapter, we will use notation
flexibly when it is convenient.  The typical unit of discussion will
be the $d$th document occurring at time $t$.  The $d$th document
discusses two countries, $c_1$ and $c_2$; these define a tuple $(\{
c_1, c_2 \}, d, t)$ (where the set $\{ c_1, c_2 \} = \{ c_2, c_1 \}$).
We will generally use $d$ to index documents, $t$ to index time, and
$c$ to index a country.  When document $d$ is given, we may refer to
its time as $t_d$ (which is unique) or to the two interacting
countries as $c_{d,1},c_{d,2}$ or $c_1,c_2$.  Alternatively, we may
refer to the documents in which a country $c$ appears as $d_{c,1},
\ldots, d_{c,D}$.  As another example, we may describe a country's
position $x_{(c_1,d,t)}$ variously as $x_{c_{d,1}}$, $x_{d,1}$, or even $x_c$
if the context is clear. Finally, the sentiment between two countries
might be variously described as $s_d$, $s_{c_1,c_2}$, $s_{d,t}$, or
$s_{c_1,c_2,d,t}$.

\subsection{Related work}

% Democracy, Political Similarity, and International Alliances, 1816-1992
% Brian Lai
% Dan Reiterb
% Department of Political Science, Emory University
Spatial models such as Item Response Theory (IRT) have been developed
over the past century by quantitative social scientists for analyzing
behavior.  While much of this work has been used to model
parliamentary voting behavior, these techniques have also been used to
model voting in the UN General Assembly. Gartzke et al., for example,
use these votes and alliance models to study the nations' affinities
\citep{gartzke:1998}.

These models have been developed for dyadic data more fully in network
models such as the latent space model \citep{hoff:2002,sarkar:2005}, in
which the probability of a link between two nodes is a function of
their latent-space distance.  The qualitative relationship of
entities' dyadic relationships has been more fully developed with text
by the relational topic model, which uses free text to model the
relationship between actors in an unsupervised setting
\citep{chang:2009}.
% Supervised topic models


% Affinity of states dataset: 
% Fading Friendships (working paper)
% Alliances, Affinities and the Activation of International Identities∗
% Erik Gartzke†
% Alex Weisiger‡
% 7 March 2011

\subsection{Inference}
We fit the \emph{MAP} objective of this probabilistic model.  This has
the benefit of both clean exposition and simple implementation, and it
can be interpreted as a form of unregularized variational inference.
We optimize the \emph{MAP} objective in this model using an
expectation maximization (EM) algorithm.

We designed the probabilistic objective to make inference tractable.
For example, the Gaussian distribution of per-document sentiment
variables $x_{c, t} | \bar x_{c,t}$ makes inference for $\bar x_{c,t}$
closed-form.  Countries' per-interaction positions $x_{c_{d,1}, t},
x_{c_{d,2},t} | \bar x_{c_{d,1},t}, \bar x_{c_{d,2}, t}, s_{c_{d1},
  c_{d2}}$ are then fit using gradient ascent.

Our goal in performing EM is to estimate parameters for the model.  Expectation allows us to find a \emph{MAP} estimate by optimizing the lower bound on the data likelihood:
\begin{align}
  \mathcal{L}_x = \log p(s_d, \bm w, \bm \beta, x)
  % & \propto p(s_d | \bm w, \beta, x, \bm \beta) \\
  & \ge \log \expectq{ \frac{q(\bar x)}{q(\bar x)}
    p(s_d | \bm w, \bm \beta, x) } \nonumber \\
  & \ge \expectq{ q(\bar x)
    \log \frac{ p(s_d | \bm w, \bm \beta, x) }{
      q(\bar x)} }
\end{align}

\subsubsection{M Step} In the M step, we estimate the mean $\bar
x_{c,t} | x_{c_1,1}, \ldots, x_{c_N,T}$ of each country's position
using a modified Kalman filter.  This step differs from a
standard Kalman filter in that we may have no or multiple observations
on any given date.\footnote{We also experimented with
  \emph{pseudo-observations} for each country at each day $t$ with
  mean 0 and variance $\sigma_p^2$.  These observations are a form of
  ``time-series regularization'' and reflect the sense that a lack of
  news is effectively neutral news.  Low values of $\sigma_p^2$ harmed
  performance, and it was best set around $10^6$.}  The prior over the
ends of the chain are standard normal.

\paragraph{Kalman updates.}
The M step seeks the expected value of the mean position $\bar x_c$
for each country $c$ given our estimate of the $D_{c,s}$ interactions
at each time $s=1, \ldots, T$:
\begin{align}
  \underset{ \expectq{ \bar x_{c,t} } }
  {\arg \max } \bar x_{c,t} | x_{c,1,1}, \ldots, x_{c,D_{c,1},1}, x_{c,D_{c,2},2},
  \ldots, x_{c,D_{c,t},T},
\end{align}
TODO(sean ): fix this.
for $t=1, \ldots, T$. The optimal value for $\bar x$ can be found with
a modified Kalman smoother \citep{kalman:1960}.  This modified Kalman
smoother requires a forward filter step and a backward filter step.
The forward filter estimates the mean position given all previous
observations (note that we use $x_{c,d,t}$ to describe the position of
country $c$ at time $t$ for interaction $d$:
\begin{align}
  \bar x_{\mbox{\tiny forth},c,t} | \bar x_{\mbox{\tiny forth},c,t-1}, \{ x_{c,d,t-1} \}_{d}
  & \gets \frac{\bar x_{\mbox{\tiny forth},c,t-1} / \sigma_{\mbox{\tiny forth},t-1}^2
    + \sum_{d=1}^{D_{c,t-1}} x_{c,d,t-1} / \sigma_{\mbox{\tiny obs}}^2}
  {1 / \sigma_{\mbox{\tiny forth},t-1}^2 + 1 / \sigma_{\mbox{\tiny obs}}^2} \\
  \sigma_{\mbox{\tiny forth},t}^2
  & \gets \frac{1}{1 / \sigma_{\mbox{\tiny forth},t - 1}^2
    + D_{c,t-1} / \sigma_{\mbox{\tiny obs}}^2} + \sigma_{\mbox{\tiny chain}}^2,
\end{align}
with initial condition $\bar x_{c,0} = 0,
\sigma_{\mbox{\tiny forth},0}^2=10$.  The backward step estimates
the chain's mean given all current and future observations:
\begin{align}
  \bar x_{\mbox{\tiny back},c,t} | \bar x_{\mbox{\tiny back,c,t+1}}, \{ x_{c,d,t} \}_d
  & \gets \frac{\bar x_{\mbox{\tiny back},c,t+1} / \sigma_{t+1}^2
    + \sum_{d=1}^{D_{c,t}} x_{c,d,t} / \sigma_{\mbox{\tiny obs}}^2}
  {1 / \sigma_{\mbox{\tiny back},t-1}^2 + 1 / \sigma_{\mbox{\tiny obs}}^2} \nonumber \\
  \sigma_{\mbox{\tiny back},t}^2
  & \gets \frac{1}{1 / (\sigma_{\mbox{\tiny back},t + 1}^2 + \sigma_{\mbox{\tiny chain}}^2)
    + D_{c,t} / \sigma_{\tiny obs}^2},
\end{align}
with initial conditions $\bar x_{\mbox{\tiny back},c,T} = 0, \sigma_{\mbox{\tiny
    backward},T}^2=10$. The smoothed means---that is, the mean of countries' positions at time $t$ given all observations for all time---are
\begin{align}
  \expectq{\bar x_{c,t}} & = \bar x_{c,t} | x \nonumber \\
  & = \bar x_{c,t} | \bar x_{\mbox{\tiny forth,c,t}}, \bar x_{\mbox{\tiny back,c,t}}, \sigma_{\mbox{\tiny back}}^2, \sigma_{\mbox{\tiny forth}}^2 \nonumber \\
  & = \frac{\bar x_{\mbox{\tiny forth},c,t} / \sigma_{\mbox{\tiny forth},t}^2
    + \bar x_{\mbox{\tiny back},c,t} / \sigma_{\mbox{\tiny back},t}^2}
  {1 / \sigma_{\mbox{\tiny forth},t}^2
    + 1 / \sigma_{\mbox{\tiny back},t}^2}
\end{align}

% x_{c,d,t}
\subsubsection{E Step} In the E-step, our goal is to infer each
nation's position $x_{c_{d,1}} | \expectq{ \bar{x}_{c,d,t} },
x_{c_{d,2}}, s_d$ during interaction $d$ given its expected mean $\expectq{ \bar
x_{c_{d,1},t_d} }$ and the text $\bm w_d$ describing this interaction,
\emph{and} given the other country's position for this interaction.
We find these positions by gradient ascent on each interaction:
\begin{align}
  & \underset{x_{c_{d,1},t}, x_{c_{d,2},t}}
  {\arg \max}
  p(\bm w_d^T \bm \beta | \mathcal{F}(x_{c_{d,1},t}, x_{c_{d,2},t}),
  \expectq{ \bar x_{c_{d,1},t} }, \expectq{ \bar x_{c_{d,2},t} }) \nonumber \\
  & = \underset{x_{c_{d,1},t}, x_{c_{d,2},t}}
  { \arg \max }
  p(\bm w_d^T \bm \beta | \mathcal{F}(x_{c_{d,1},t}, x_{c_{d,2},t}),
  \bm \beta)
  p(x_{c_{d,1},t} | \expectq{ \bar x_{c_{d,1},t} } )
  p(x_{c_{d,2},t} | \expectq{ \bar x_{c_{d,2},t} } ).
\end{align}

\subsection{Empirical studies: comparisons with ground truth}
We now turn to an experimental analysis of this model.  We first
describe the two label types we have used to define sentiment $s_d$
within this model and the new archive to which we fit this model.  We
then evaluate the model's ability to infer the relationships between
countries and compare results from models inferred with the two
different label types.

\subsection{News archive and tokenization}

\paragraph{New York Times.}
We fit and evaluate this model over news articles discussing 245
nations and territories from twenty years of the \emph{New York Times}
(NYT).  This collection spanned the years 1987 to 2007, a period which
included both Gulf wars; the collapse of the Soviet Union; the
reunification of Germany; September 11th, 2001; and countless other
world events.

\paragraph{Data preparation.}
We used articles from the Foreign, Business, Financial, and Magazine
desks of the newspaper during this period. We made an important
assumption that the scope of foreign sentiment discussion is at the
level of a paragraph.  We therefore used the subset of paragraphs
which discuss exactly two nations as ``documents'' $d$, a collection
of 257,472 paragraphs from 1987 to 2007.  We then defined a vocabulary
to be those words which appeared at least twenty times in the
collection, in no more than 40\% of documents, and in at least 0.1\%
of documents.

This resulted in a vocabulary of 5958 words, mentioned by 40,356
paragraphs. We randomly selected 80\% (32,249) distict paragraphs from
this set as training examples and used the remaining examples to
evaluate our model.

\subsection{Coding sentiment}
\label{section:sentiment_models}

We next labeled our training examples with both lay reviewers and
expert labels, representing opposite ends of the human-label spectrum.

\subsubsection{Novice labels: Amazon Mechanical Turk ratings}
\label{section:mturk}

\emph{Amazon Mechanical Turk} (AMT) is a crowdsourcing platform which
provides a \emph{requestor} (the author of this thesis) with access to
thousands of \emph{workers} who perform simple tasks over the
Internet.  Although the requestor can use tests to ensure that workers
are high-quality, these workers are typically not experts.

\begin{figure*}
  \setlength\fboxsep{0pt}
  \setlength\fboxrule{0.5pt}
  \center \fbox{\includegraphics[width=1.0\textwidth]{chapter_foreign_relations/figures/mturk_screenshot.png}}
  \label{fig:mechanical_turk_sample}
  \small\caption{A screenshot of a Mechanical Turk labeling task.
    Sometimes relationships may be complicated; both raters gave this
    example a score of ``slightly positive''.}
  \normalsize
\end{figure*}

To fit the model, we asked \emph{Amazon Mechanical Turk} workers to
rate the sentiment between two nations mentioned in the text of a
paragraph on the scale -5 (mortal enemies), $\ldots$, 5 (very good
relationship). We illustrate a rating task (as seen by a Mechanical
Turk worker) in \myfig{mechanical_turk_sample}. Raters were asked to
review a random subset of 3607 paragraphs from the original
collection.  Before fitting the model, we manually disqualified eight
raters (out of 85) who consistently performed poor ratings.

With all rated paragraphs which were not in the test set, we fit the
coefficients $\bm \beta$ of the text regression discussed in
Section~\ref{section:model}.  This coefficient was then treated as
constant in the joint model in Figure~\ref{fig:fa_gm} to allow us to
infer sentiment from the words of all 32,249 training paragraphs.  We
illustrate the $\bm \beta$ inferred from Mechanical Turk-labeled
paragraphs in \myfig{fr_example_betas} (left).

\subsubsection{Expert labels: Correlates of War}
\label{section:correlates_of_war}

We also used a combined set of expert labels based on the Correlates of War \citep{sarkees:2010} and Issue Correlates of War \citep{hensel:2001}.
\begin{itemize}
  \item The \emph{Correlates of War} project ``seeks to facilitate the
    collection, dissemination, and use of accurate and reliable
    quantitative data in international relations''
    \citep{cow_webpage:2012}.  The project provides labels describing the
    relationships between pairs of countries from 1823 to 2003.
    At-war is a binary relationship (either countries are at war, or
    they are at peace). We used a list of CoW inter-state wars
    (version 4.0) from 1823 to 2003
    \citep{sarkees:2010}.
  \item The \emph{Issue Correlates of War} project ``is a research
    project that is collecting systematic data on contentious issues
    in world politics'' \citep{icow_webpage:2012}, and they provide
    expert labels on a variety of inter-state conflicts that \emph{do
      not require militarized conflict}.  However, these issue label
    do require documented evidence of contention between states; such
    issues include maritime and territorial disputes
    \citep{icow_webpage:2012,hensel:2001}. The Issue Correlates of War
    are not part of the same project (or produced by the same
    researchers) as the Correlates of War.
\end{itemize}

We combined the datasets by treating two countries as having a
rating of -5 if they are at war from the Correlates of War codes and
-1 if there is any contentious issue between the countries in the
Issue Correlates of War.\footnote{These values were selected to
  correspond roughly to the Mechanical Turk labels.}.  All other pairs
of countries were treated as having a rating of 0.1.  As before, we
fit the text regression parameters $\bm \beta$ using these labels on the
training set and evaluated countries' ratings on the test dataset. We
illustrate $\bm \beta$ fit to CoW-labeled paragraphs in
\myfig{fr_example_betas} (left).

\begin{figure}
  \begin{tabular}{cc}
    \includegraphics[width=0.4\textwidth]{chapter_foreign_relations/figures/mturk_sample_words.pdf} &
    \includegraphics[width=0.4\textwidth]{chapter_foreign_relations/figures/cow_sample_words.pdf} \\
    \end{tabular}
  \caption{Coefficients $\bm \beta_w$ for selected words $w$ fit on text
    labeled by Amazon Mechanical Turk workers (left) and Correlates of
    War data (right). Coefficients fit from Mechanical Turk labels are
    more clearly separated than those fit to Correlates of War labels;
    this is likely due to explicit positive sentiment in that dataset.
    The $x$-axis is $\bm \beta$, and the $y$-axis is used for display (it
    corresponds to no variable).  Size of each word is proportional to
    $\sqrt{\mbox{frequency}}$, and color corresponds to $\bm \beta$.}
  \label{fig:fr_example_betas}
\end{figure}

\subsubsection{Casual vs. expert labels}
The CoW represent a data source which is modestly related to
Mechanical Turk ratings. In the NYT dataset, CoW ratings and
Mechanical Turk ratings were correlated at $\sigma=0.196$.  We can
better understand the difference between these ratings with some
examples of paragraphs $d$ having high and low inter-country sentiment
$s_d$:
\begin{itemize}
  \item AMT $= 1$, CoW $=-5$: \begin{quote} \emph{As an
    indication of the dangers the damage occurred in waters where
    military spokesmen said no mines had been suspected before but
    where a \textbf{Saudi} officer said today that some 22 were later
    found. \textbf{Iraq}i mines widely deployed [sic]} (February 1991)
    \citep{cushman:1991}
\end{quote}
  \item AMT $ = -5$, CoW $=0.1$: \begin{quote}\emph{Not since
    the grim old days of the cold war have relations between the
    \textbf{United States} and \textbf{Russia} been quite as problematic as they
    are this weekend on the eve of president clinton's visit for
    celebrations marking the 50th anniversary of the allied victory in
    europe in World War II.} (May 1995) \citep{apple:1995}
\end{quote}
\end{itemize}

The first of these examples outlines a limitation in our modeling
assumptions: a single paragraph is sometimes too small a unit of
discussion.  In this example, Mechanical Turk workers likely missed
the larger context of the article about the Gulf war (including the
article's title, \emph{War in the Gulf: Sea Mines; Allied Ships Hunt
  Gulf for Iraqi Mines}).

The second example represents a limitation of both data sources.  The
two Mechanical Turk ratings of -5 were clearly too strong, as the
countries are not at war; but AMT workers likely based their rating in
part on the reference to World War II (the instructions provided to
MTurk workers suggest that a rating of -3--possibly -1--would be more
appropriate).  In 1995, the United States and Russia were not at war
and had no documented territorial conflicts.  This means that this
sentiment was not reflected in the CoW labels -- which defaulted to
0.1.

\label{section:experiments}

\subsection{Quantitative results}

\begin{figure}
%%   \begin{tabular}{|c|c|c|c|c|c|c|c|}
%%    \hline
%%   Link $\mathcal{F}(\bm x_{c_1}, y_{c_1}, \bm x_{c_2}, y_{c_2})$ & & & & & & \\
%%   \hline
%%   \textbf{Dimension of $\bm x$} & 0 & 1 & 2 & 3 & 4 & 5 & 6 & 7 & 8 & 9 \\
%%   \hline
%%   $y_{c_1} + y_{c_2}$ & 0.0964 & - & - & - & - & - & - & - & - & - \\
%%   \hline
%%   $\bm x_{c_1}^T \bm x_{c_2}$
%%   & 0.1036 & 0.1013 & 0.1050 & 0.1038 & 0.1033 & 0.1025 & 0.1018 & & & \\
%%   \hline
%%   $y_{c_1} + y_{c_2} + \bm x_{c_1}^T \bm x_{c_2}$
%%   & 0.0964 & 0.0943 & 0.0940 & 0.0936 & 0.0935 & 0.0934 & 0.0934 & & & \\
%%   \hline
%%   $-\log(||\bm x_{c_1} - \bm x_{c_2}||_2^2 + 1)$
%%   & 0.1036 & 0.1037 & 0.0978 & 0.0957 & 0.0951 & 0.0947 & 0.0943 & & & \\
%%   \hline
%%   $y_{c_1} + y_{c_2} -\log(||\bm x_{c_1} - \bm x_{c_2}||_2^2 + 1)$
%%   & 0.0964 & 0.0949 & 0.0943 & 0.0938 & 0.0939 & 0.0937 & 0.0936 & & & \\
%%   \hline
%%   mean($\{s_d\}_d$) & 0.1036 & - & - & - & - & - & - & & & \\
%%   \hline
%%   \end{tabular}
%%   \vspace{30pt}
%%   \begin{tabular}{|c|c|c|c|c|c|c|c|}
%%    \hline
%%   Link $\mathcal{F}(\bm x_{c_1}, y_{c_1}, \bm x_{c_2}, y_{c_2})$ & & & \\
%%   \hline
%%   \textbf{Dimension of $\bm x$} & 0 & 1 & 2 & 3 & 4 & 5 & 6 \\
%%   \hline
%%   $y_{c_1} + y_{c_2}$ & 0.0964 & - & - & - & - & - & - \\
%%   \hline
%%   $\bm x_{c_1}^T \bm x_{c_2}$
%%   & 0.1036 & 0.1013 & 0.1050 & 0.1038 & 0.1033 & 0.1025 & 0.1018 \\
%%   \hline
%%   $y_{c_1} + y_{c_2} + \bm x_{c_1}^T \bm x_{c_2}$
%%   & 0.0964 & 0.0943 & 0.0940 & 0.0936 & 0.0935 & 0.0934 & 0.0934 \\
%%   \hline
%%   $-\log(||\bm x_{c_1} - \bm x_{c_2}||_2^2 + 1)$
%%   & 0.1036 & 0.1037 & 0.0978 & 0.0957 & 0.0951 & 0.0947 & 0.0943 \\
%%   \hline
%%   $y_{c_1} + y_{c_2} -\log(||\bm x_{c_1} - \bm x_{c_2}||_2^2 + 1)$
%%   & 0.0964 & 0.0949 & 0.0943 & 0.0938 & 0.0939 & 0.0937 & 0.0936 \\
%%   \hline
%%   mean($\{s_d\}_d$) & 0.1036 & - & - & - & - & - & - \\
%%   \hline
\begin{tabular}{m{2cm}cc}
  \tiny{Correlates of War} \vspace{10pt} & \includegraphics[height=0.25\textheight]{chapter_foreign_relations/figures/008_static_model_results.pdf} & \includegraphics[height=0.25\textheight]{chapter_foreign_relations/figures/009_dynamic_model_results.pdf} \\
  \tiny{Mechanical Turk} \vspace{10pt} & \includegraphics[height=0.25\textheight]{chapter_foreign_relations/figures/010_static_model_results.pdf} & \includegraphics[height=0.25\textheight]{chapter_foreign_relations/figures/010_dynamic_model_results.pdf}
  \\ & Static & Dynamic (Interaction over time) \\
\end{tabular}
  \label{fig:fr_supervised_performance}
   \caption{The dyadic sentiment model captures text well . Each
     colored line represents performance of the supervised model on a collection
     of heldout documents across twenty years of New York Times
     articles. An inner-product model with four dimensions (plus
     intercepts) performs well for most settings.  A distance model
     with many dimensions but no intercepts also performs well across
     a range of assumptions, performing best with many dimensions.}
\end{figure}

\paragraph{Prediction.}
For two nations $c_1$ and $c_2$ mentioned together at time $t$, we
predict their sentiment to be $\tilde s_{c_1, c_2} = \bar x_{c_1,t} \bar x_{c_2, t}$.

\paragraph{Static latent space.}
We first evaluate the latent-space assumption that countries'
interactions can be summarized by their positions in a latent space.
For this, we used a simplified model of sentiment, in which documents'
mean positions $\bar x_{c_1}, x_{c_2}$ do not change over time,
effectively setting their chain variance $\sigma_{\mbox{\tiny
    chain}}^2$ to zero.  The
sentiment between countries interacting in document $d$ is $s_d =
\mathcal{F}(x_{c_1}, x_{c_2})$, with $x_c \sim \mathcal{N}(\bar x_c,
\sigma_D^2)$.

We tested this assumption for five link functions $\mathcal{F}(\bm
x_{c_1}, \bm x_{c_2})$.  We summarize these functions in Table~\ref{table:fr_link_functions}.
\begin{figure}
\center
\begin{tabular}{|l|l|m{3.9cm}|}
      \hline
      Description & $\mathcal{F}(\bm x_1, \bm x_2)$ & where ... \\
      \hline
      distance & $-\log(|| \bm z_{1} - \bm
      z_{2} ||_2^2 + 1)$ & $\bm z_{1} = \bm x_{1,2:D}, \bm z_2 = \bm x_{1,2:D}$ \\
      \hline
      inner product & $\bm z_{1}^T \bm z_{2}$ & $\bm z_{1} = \bm
      x_{1},$ $\bm z_2 = \bm x_{2}$ \\
      \hline
      intercept & $y_1 + y_2$ & $y_1 = x_{1,1}, y_2 = x_{2,1}$ \\
      \hline
      intercept / inner product & $y_1 + y_2 + \bm z_{1}^T \bm
     z_{2}$ & $y_1 = x_{1,1}, y_2 = x_{2,1},$ \\
    & & $\bm z_{1}
     = \bm x_{1,2:D}, \bm z_2 = \bm x_{2,2:D}$ \\
      \hline
     intercept / distance & $y_1 + y_2 - \log(|| \bm z_{1} - \bm
     z_{2} ||_2^2 + 1)$ & $y_1 = x_{1,1}, y_2 = x_{2,1},$ \\
     & & $\bm z_{1}
     = \bm x_{1,2:D}, \bm z_2 = \bm x_{2,2:D}$ \\
     \hline
    \end{tabular} \\
\label{table:fr_link_functions}
\caption{Link functions $\mathcal{F}: \mathbb{R}^P \times \mathbb{R}^P
  \rightarrow \mathbb{R}$.  Intercept link functions introduce
  per-country intercepts that indicate how prone a country is to war;
  distance link functions are based on the distance between countries'
  vectors; and inner-product link functions represent sentiment as a
  function of countries' political ``orientations''.}
\end{figure}

We measured the model's ability to predict inferences from the
text-based sentiment model $s_d = \bm w_d^T \bm \beta$ for $1 \le
\mbox{dim}(\bm z) \le 9$ and report the mean-squared-error (MSE) of
these predictions in \myfig{fr_supervised_performance}.  We find that
the the inner-product assumption $\bar z_{c_1}^T \bar z_{c_2}$ alone
is poor because it provides no natural natural way to model countries that are
in conflict with many other countries (to model this, we
would require $\bar x_c$ for that country to have large magnitude,
and for other countries to have $\bar x_c$ with large magnitude
but opposite direction; this becomes difficult without requiring many
dimensions).

If the inner-product model is endowed with the intercepts $y_{c_1},
y_{c_2}$, however, its predictive performance increases substantially.
Both the \verb!intercept / inner product! and
\verb!intercept / distance! link functions perform very well, because
they can use intercepts to explain how conflict-prone a country is;
and they can use $\bm \bar z$ to explain how each country interacts
with others.  Interestingly, the \verb!distance! link function is able
to model data well with many dimensions without an indication that the
model overfits (we only measured this up to 9 dimensions).

\paragraph{The benefit in adding a time-series assumption.}

We can add more flexibility to this model -- and an ability to model
much more interesting behavior -- by adding a time dimension, allowing
$\bar x_c$ to drift over time for each country $c$.  We illustrate
these results in \myfig{fr_supervised_performance}.

The inner product model again performed poorly, often worse than the
baseline model.  Adding an intercept term harms performance for the
intercept model.  The time-series assumption overall improved
performance for correlates of war and harmed performance for
Mechanical Turk labels.

We note that the time-series models performed better than the static
model for the CoW labels but \emph{not} for the Mechanical Turk
labels.  One theory is that the formal relationships between countries
-- as determined by expert labels -- is indeed changing over time;
while the lay relationships between these countries -- as determined
by lay interpretations of countries' relationships -- remains more
static over time.

\paragraph{Improvement due to zero-reversion regularization}
A further explanation for the decrease in performance for the
time-series models (compared to the static model) is sensitivity to
parameters.  The static models have one parameter for each link
function: the prior of countries' positions $\sigma_{c}^2$.  In the
dynamic model, we must set the priors over countries' positions
$\sigma_{c,d}^2$ for each interaction, chain variance
$\sigma_{\mbox{\tiny chain }}^2$, and zero-reversion variance
$\sigma_{p}^2$.  We selected chain variance $\sigma_{\mbox{\tiny
    chain}}^2=0.0001$ and zero-reversion variance $\sigma_p^2=1,0.01$
by grid search for these models at $3$ dimensions.

\paragraph{Static latent space.}
What relationships between nations does this model infer?  Because the
relationships between countries are treated as functions of their
positions $\bm x \in \mathbb{R}^P$, we can interpret these countries'
positions $\bm x$ as summaries of countries' geopolitical
orientations.  We illustrate the positions of selected countries in
\myfig{fr_intercept_distance_positions}.

\begin{figure}
  \begin{tabular}{cc}
    \includegraphics[width=0.5\textwidth]{chapter_foreign_relations/figures/011_static_positions_mturk.pdf} &
    \includegraphics[width=0.5\textwidth]{chapter_foreign_relations/figures/011_static_positions_cow.pdf}
    \\
    Mechanical Turk & Correlates of War \\
  \end{tabular}
  \caption{Positions of selected countries according to the static
    issue-adjusted model for articles labeled with Amazon Mechanical
    Turk (left) and Correlates of War (right).  Countries' positions
    were inferred with the intercept / distance model, with distance
    dimension $P$=2.  Intercepts are illustrated by color.}
  \label{fig:fr_intercept_distance_positions}
\end{figure}

With both CoW and AMT labels, the relationships between
countries can be inferred from the distance between their positions.
In \myfig{fr_intercept_distance_positions}, the United States stands
out from a cluster of other countries, with Iraq, Iran, and
Afghanistan--countries with which the U.S. has been at odds in the past
twenty years--furthest away.

Correlates of War and Mechanical Turk labels provide different
patterns of inter-country sentiment.  Countries' positions under CoW
tend to be very clustered, with a few outliers, while their positions under
AMT labels are more uniformly distributed.  However, the two datasets
provide extraordinarily consistent measures of countries' relationships.

To measure the consistency of these two models, we measured the
Spearman rank correlation coefficient 
\begin{align*}
\underset{c_1, c_2 \in C, c_1   \neq c_2}
{\operatorname{\mbox{Correlation}}}(d_{\mbox{\tiny AMT}}(c_1, c_2), d_{\mbox{\tiny CoW}}(c_1, c_2))
\end{align*}
between all pairs of countries $c_1, c_2$. The two-dimensional
\verb!intercept / distance! models have a spearman rank correlation
coefficient of $\sigma=0.900$.  Of course, these $|C| \choose{2}$
distances are far from independent, and a single outlier in each model
could skew the correlation.  To mitigate any such effect, we also measured
the average correlation coefficient
\begin{align*}
  \frac{1}{|C|} \sum_{c_1 \in C}
  \underset{c_2 \in C, c_2 \neq c_1}
  {\operatorname{\mbox{Correlation}}}(d_{\mbox{\tiny AMT}}(c_i, c_2), d_{\mbox{\tiny
        CoW}}(c_i, c_2) ),
\end{align*}
which was \emph{even higher}, at $\sigma=0.901$. Recall that this is
higher than the correlation coefficient between the original labels
($\sigma=0.196$) -- an effect possible possible because these models
remove noise.

Under the second metric of correlation, most per-country correlations
were very high: over 90\% of countries had correlation coefficient
higher than 0.86.  One of the most-differently-represented countries
in this collection under the two different models was Iran, which
accounted for 7\% of documents; the per-Iran correlation coefficient
$\underset{c_2 \in C}{\operatorname{\mbox{Cor}}}(d_{\mbox{\tiny AMT}}(\mbox{\small Iran}, c_2),
d_{\mbox{\tiny CoW}}(\mbox{\small Iran}, c_2)$ was 0.65 (higher only
than Eritrea, which was 0.62 but accounted for 0.2\% of documents).

\paragraph{Mutual sentiment with the United States and differences
 between CoW and AMT model fits.}
We illustrate mutual sentiment with the United states for a selection
of these countries over time in \myfig{country_positions_over_time}.
To estimate the sentiment in these plots, we fit the
\verb!intercept / distance! model with $\mbox{dim}(\bm z) = 2$.
We summarize major events for two of these countries below.
\begin{itemize}
  \item \textbf{Ukraine} was emancipated in 1991 with the dissolution
    of the Soviet Union.  The U.S. has given Ukraine over $4.1$
    billion in aid, targeted to ``promote political, security, and
    economic reform and to address urgent social and humanitarian
    needs'' \citep{ukrainestate:2012}.  In return, Ukrain has been an
    active member of the UN and has assisted the NATO allies with
    defense aid in Kosovo (1999), Afghanistan (2011), Iraq, the Middle
    East, and Africa.  Ukraine adopted its first post-Soviet
    constitution June 28, 1996, the same year taking part in the
    Olympics for the first time as an independent nation (the Olympics
    were hosted in the U.S. that year).  At the same time, Ukraine has
    been taking acrive steps in eliminating the nuclear weapons
    program it inherited, permanently closing the last operating
    reactor at the Chornobyl site in 2000 \citep{ukrainestate:2012}.

    Ukraine's sentiment with Iraq, as inferred from the AMT model, was
    at its lowest in January 1993, January 1998, and again in April
    2006.  Its CoW sentiment with Iraq was at its lowest in May 2006,
    April 2003, and February 1991 (technically before its
    independence, during the Persian Gulf War). Its relationship with
    the U.S. was much stronger than with Iraq, peaking in 1996 (AMT)
    and June 2002 (CoW), when it supported the U.S. invasion of Iraq.

  \item \textbf{Iran} has had a poor relationship with the United
    States since the U.S. Embassy seizure in 1981.  Between 1987 and
    1988, U.S. and Iranian forces clashed in the Persian Gulf.
    \citep{irancia:2012}.  Transfers of power have since then increased
    political tension, with the election of a reformist president in
    1997 and a reformist legislature in 2000, followed by conservative
    re-elections starting in 2003 and continuing through 2004.
    Hardliner President President Mahmud Ahmadinejad was inaugurated
    in August 2005 and re-elected in 2009 \citep{irancia:2012}.

    Ahmadinejad's rule has been met with increasing pressure from
    United Nations countries.  The Council has made successive
    resolutions imposing santions on Iran in 2006, 2007, 2008, and
    2010 \citep{iranstate:2012}.

    The Mechanical Turk sentiment between Iran and the U.S. has
    clearly dropped in the lead-up to Ahmadinejad's election (see
    again \myfig{country_positions_over_time}), but this contrasts
    with the Correlates of War sentiment, which was lowest in 1988,
    when AMT sentiment was not as low.

\end{itemize}
 
Both of these low periods with Iran are clearly periods of bad
relationships between these countries, but why did one model pick up
sentiment in one case and not the other?  This could be explained in
part because the tension picked up by the CoW labels was unilateral,
while the tension picked up in the later period did not fall under the
dictum of CoW labels: the U.S. and Iran were neither at war nor having
a territorial dispute.  Instead, the U.S., as a member of the U.N. has
supported Iran sanctions.


\begin{figure}
  \center
    \includegraphics[width=1\textwidth]{chapter_foreign_relations/figures/012_fr_cow_mutual_sentiment_with_us.pdf}
    \includegraphics[width=1\textwidth]{chapter_foreign_relations/figures/012_fr_mturk_mutual_sentiment_with_us.pdf}
  \label{fig:country_positions_over_time}
  \caption{Selected countries' relationship with the United States
    over time.  Each line in the plot above represents a specific
    country's relationship with the United states inferred with the
    intercept/distance link function, with a two-dimensional distance
    space, using CoW labels (top) and AMT labels (bottom).  Sentiment
    between all countries and either Iran or Pakistan was
    least consistent between CoW and AMT. Ukraine was
    the most consistently represented with these labels.}
\end{figure}

% > a = read.csv("../../data/v4/v4-doc-training_samples.csv", as.is=TRUE, header=FALSE)

  %  We then evaluate perplexity
 % \begin{align}
 %   \mbox{perp}_d & = \mathcal{E_{\hat D}} \log p(w | \bar x_{d_{c1}}
 %     \bar x_{d_{c2}} ) \\
 %     & = \frac{1}{N} \sum_N \sum_{W_{d_n}}
 %     \log p(w | \bar x_{d_{c1}} \bar x_{d_{c2}} ) \\
 % \end{align}

% \subsubsection*{Bivariate change point detection}
% In addition to identifying the latent positions of nations over
% time, we can pinpoint periods of great change or upheaval.  Sudden
% changes in a nation's position is an indication of newsworthy events
% in its history; simultaneous changes in two nations' positions is an
% indication that they are both taking part in these newsworthy events.

% The task of identifying sudden changes in a time-series is known as
% change point detection.  Change point detection is frequently
% addressed by simple univariate significance tests.  We consider
% changes in the statistic $\bar s_{c1,c2} := \bar x_{c1} \bar x_{c2}$
% (see Equation~\ref{figure:sentiment}).  Because $\bar s_{c1,c1}$ may
% be change significantly when either $x_{c_1}$ or $x_{c_2}$ changes, we
% search for simultaneous significant changes in $\bar x_{c_1}$, $\bar
% x_{c_2}$, and $\bar s_{c1,c2}$ (Note that $\mathcal{E}[s_{c1,c2}] \neq
% \bar s_{c1,c2}$; we compute $\bar s_{c1,c2}$ out of convenience.

