\section{Conclusions}

In this chapter we took a closer look at the story within a collection
of documents.  To do this, we reviewed a model for representing the
relationship between countries, and we saw that this model provides an
empirically meaningful benefit over simpler baselines.  We also
demonstrated that the predicted sentiment between pairs of countries
with two entirely different sets of labels was strikingly similar.  We
finally demonstrated that an unsupervised model can produce a
sentiment dimension aligned with our conception of inter-nation
sentiment.

The set of assumptions we used in this chapter provide a broad view
of global politics.  Unfortunately it provides no sense for the
internal factors motivating the positions countries take within the
latent space.  In the following chapter we will zoom in to take a
closer look at how politicians within a country---the United States in
particular---make decisions.  To do this, we will use the text of the
bills on which they are voting to better understand the positions they
take.  By using the text of bills, we will also overcome some
limitations of a traditional model of how lawmakers vote.

We will continue to see two of the primitives discussed in this
chapter.  The traditional model of how lawmakers vote is in fact very
much like the latent-space model we described in this chapter, and
lawmakers' positions within this latent space are widely disseminated
statistics. Second, we will continue to see that tools for text
analysis -- both mixed-membership models and text regression -- can
provide meaningful extensions of this model.
