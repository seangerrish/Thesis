\subsection*{A comparison with unsupervised relationship mining}

% There are some limitations to modeling sentiment in a supervised way.
%   - Don't have explicit sentiment labels
%   - The interactions between countries is better described
%     by a dimension orthogonal to "war".
%     - Trade
%     - Culture and society

The preceding approach has limitations, of course.  First, sentiment
labels measure only one kind of interaction: whether countries are at
war or peace.  In reality, relationships between countries may be
characterized in many ways, some of which are independent of the $[
war, peace ]$ dimension.  For example, the relationship between coutries
may be characterized by trade in goods, or by the exchange of culture
and society.  Another limitation to a supervised sentiment model is
that labels of the sentiment between countries may be unavailable or
limited, or (as we saw before), the labels may be noisy.

% in this section, we will describe a model which will allow us to discover
% unsupervised relationships
%   - We will use a modul much like RTM
%   - RTM is ...
In this section we will briefly compare the results of the previous
model with the results of an unsupervised model.  Because the
unsupervised model is preliminary, we leave details of it in
\myapp{app_unsupervised_sentiment}.  The unsupervised sentiment model
uses the same latent-space assumtion that we introduced in
Sections~\ref{sec:fr_latent_space_model}
and~\ref{sec:fr_time_series_model}.  The curious reader can refer to
the appendix for a fuller description of these assumptions.

\subsubsection*{Topics}
A key assumption behind the unsupervised sentiment model is that each
document can be described by a mixture of four topics.  Two of these
topics correspond to the two countries discussed in the paragraph
(there are $C$ of these topics, one for each country).  A third topic
is a ``background'' topic, and the fourth topic is one of two
sentiment topics.

By fitting the unsupervised model, we learn which words are
most-likely in each of these topics.  With these assumptions, we
inferred a set of topics using the same NYT corpus for the supervised
sentiment model.  \mytab{fr_unsupervised_topics} lists the most-likely
words from sample of topics fit to these twenty years of articles.

\paragraph{State-specific topics $\beta_{C,\cdot}$.}  The
state-specific topics describe words used when one of these countries
is mentioned in text.  Many of these topics are intuitive: ``oil''
shows up in the Iran topic, and ``drug'' is the top word in the Mexico
topic.  As these countries are mentioned in a major U.S. news source,
the topics are sometimes biased toward ideas specific to the
U.S. relationship with these countries (for example, ``border'' and
``traffickers'' in the Mexico topic).  The U.S. topic contains phrases
specific to policy and leadership.

While these topics are intuitive, they serve little role in analyzing
this collection.  From a modeling perspective, they serve as a
``sponge'' to explain away words commonly used to describe a country,
especially when those words might otherwise be interpreted to refer to
a specific relationship.

\paragraph{An economics/military dichotomy.}
The sentiment topic, on the other hand, appears to demonstrate that
one of the most prominent directions of variance in the text of
paragraphs corresponds to the sentiment that we have been measuring.

Again using the convention that $\kappa_d=1$ indicates \emph{negative}
sentiment between countries, the negative-sentiment topic
$\beta_{S,1}$ matches our intuition: it contains words typically
associated with conflict: \emph{military, officials, soldiers, killed,
  troops}, and \emph{police} are among the top words. On the other
hand, the words most likely in the supplementary topic $\beta_{S,0}$
are associated more with economics: \emph{million, percent, people,
  billion, oil}, and \emph{officials}.

Are these words the same words that tend to be associated with expert
labels of sentiment?  The answer is a definitive yes.  To quantify this, we used
coefficients from the text regression fit to Mechanical Turk ratings
in the last section.  Among the top 12 words in this topic (shown in
\mytab{fr_unsupervised_topics}), we estimated the average coefficient
learned in the supervised sentiment model.  The average coefficient
for these terms was -0.225, which is less than the mean 0.007 of the
entire vocabulary ($p < 0.02$ by a 2-sample $t$-test).  The mean of
the per-word sentiment $\beta$ for this collection of words was at the
$20th$ percentile of words in the vocabulary.

This contrasts with the top words in either the complementary topic
$\beta_{S,0}$ or the background topic $\beta_{B}$.  The top words in
these topics had respective mean sentiments $\beta$ of $-0.11, -0.08$.
Neither of these was statistically noteworthy ($p=0.15,0.16$).

% million & military \\
% percent & officials \\
% people & soldiers \\
% billion & killed \\
% oil & troops \\
% officials & police \\
% country & forces \\

% What do the topics look like?
%  background topic?

\begin{figure}
%         Country
% 2   united_states
% 3            iran
% 4           japan
% 8          canada
% 13          china
% 17           iraq
% 24        ireland
% 31  great_britain
% 243            X1
%                                                                                             Topics
% 2   officials,military,official,policy,political,support,government,meeting,leaders,administration
% 3                             nuclear,war,program,weapons,officials,arms,oil,hostages,gulf,uranium
% 4               trade,economic,countries,officials,market,economy,world,military,companies,markets
% 8                      trade,country,free,province,border,agreement,speaking,percent,law,officials
% 13               rights,human,trade,relations,officials,nuclear,visit,political,democracy,economic
% 17               forces,weapons,military,troops,officials,government,security,war,chemical,country
% 24                     police,people,province,peace,killed,political,bomb,control,violence,killing
% 31          officials,government,authorities,intelligence,week,people,minister,colony,time,country
% 243             war,political,officials,country,people,military,international,peace,confirmed,week
% mexico: drug,officials,border,law,enforcement,traffickers,agents,police,authorities,trade,cocaine,illegal
  \center
\begin{tabular}{|c|c|}
  \hline
  \textbf{Background topic ($\beta_{B}$)} \\
  \hline
  war \\
  political \\
  officials \\
  country \\
  people \\
  military \\
  international \\
  peace \\
  confirmed \\
  week \\
  following \\
  government \\
  \hline
\end{tabular}
\hspace{30pt} \begin{tabular}{|cc|}
  \hline
  \textbf{Economics topic ($\beta_{S,0}$) vs.} &
  \textbf{Military topic ($\beta_{S,1}$)} \\
  \hline
  million & military \\
  percent & officials \\
  people & soldiers \\
  billion & killed \\
  oil & troops \\
  officials & police \\
  country & forces \\
  money & people \\
  aid & attack \\
  government & border \\
  companies & near \\
  military & air \\
  \hline
\end{tabular}
\\
\vspace{30pt}
\begin{tabular}{|c|c|c|}
  \hline
  \textbf{Pakistan ($\beta_{C,\mbox{\tiny Pakistan}}$)} &
  \textbf{Mexico ($\beta_{C,\mbox{\tiny Mexico}}$)} &
  \textbf{Israel ($\beta_{C,\mbox{\tiny Israel}}$)} \\
% Iran: nuclear,war,program,weapons,officials,arms,oil,hostages,gulf,uranium
% China: rights,human,trade,relations,officials,nuclear,visit,political,democracy,economic
% india: nuclear,countries,border,weapons,tests,talks,nations,military,state,territory,war,militants
% pakistan: nuclear,weapons,military,officials,terrorism,border,war,government,aid,support,countries,militants
% mexico: drug,officials,border,law,enforcement,traffickers,agents,police,authorities,trade,cocaine,illegal
% israel: peace,territories,occupied,talks,officials,negotiations,agreement,state,settlement,security,settlements,violence
  \hline
  nuclear & drug & peace \\
  weapons & officials & territories \\
  military & border & occupied \\
  officials & law & talks \\
  terrorism & enforcement & officials \\
  border & traffickers & negotiations \\
  war & agents & agreement \\
  government & police & state\\
  aid & authorities & settlement \\
  support & trade & security \\
  \hline
\end{tabular}
\vspace{10pt}
\begin{tabular}{|c|c|c|}
  \hline
  \textbf{United States ($\beta_{C,\mbox{\tiny United States}}$)} &
  \textbf{Iran ($\beta_{C,\mbox{\tiny Iran}}$)} &
  \textbf{China ($\beta_{C,\mbox{\tiny China}}$)} \\
% Iran: nuclear,war,program,weapons,officials,arms,oil,hostages,gulf,uranium
% China: rights,human,trade,relations,officials,nuclear,visit,political,democracy,economic
% india: nuclear,countries,border,weapons,tests,talks,nations,military,state,territory,war,militants
% pakistan: nuclear,weapons,military,officials,terrorism,border,war,government,aid,support,countries,militants
% mexico: drug,officials,border,law,enforcement,traffickers,agents,police,authorities,trade,cocaine,illegal
  \hline
  officials & nuclear & rights \\
  military & war & human \\
  official & program & trade \\
  policy & weapons & relations \\
  political & officials & officials \\
  support & arms & nuclear \\
  government & oil & visit \\
  meeting & hostages & political \\
  leaders & gulf & democracy \\
  administration & uranium & economic \\
  \hline
\end{tabular}
\caption{Per-country topics ($\beta_{C,\cdot}$), a background topics ($\beta_{B,0}$), and the two interaction topics ($\beta_{S,0}, \beta_{S,1}$).}
\label{fig:fr_unsupervised_topics}
\end{figure}
