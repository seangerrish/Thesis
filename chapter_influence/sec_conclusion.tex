
% \section{Discussion}

\section{Conclusions}

Traditional bibliometrics like citations are widely used for
understanding collections of text documents.  Much of the past work
for identifying influential documents focuses on measuring or
predicting citations for corpora which have citations.  In this
chapter we described the DIM, which is developed for time-series
corpora without bibliometrics.  We have demonstrated measured
consistency with citations with the model, controlling for confounders
like document length.  However, the information provided by the model
transcends this: the influence score has anecdotally been demonstrated
to provide qualitatively different information than citations.

Based only on the changing statistics of the language in a corpus, we
computed a measure of influence that is significantly related to
observed citation counts. That said, it would be useful to better
understand how this metric is qualitatively different from citations
and other bibliometrics: expert judgment or usage information obtained
from digital libraries might be some avenues.  We leave this for
future work.

We considered several documents evaluated by the model:
\cite{brown:1993} and \cite{toole:1984}, which both had high citations
and high posterior influence; and \cite{marcus:1993}, which had high
citations and low posterior influence.
%posterior influence; and \cite{Nature.success:1969}, which had few or
%no citations and high posterior influence. These examples illustrate
%that the model can produce results
%different from citations which are still interpretable and meaningful
%to researchers: while citation counts often point to resources based
%on new or traditional ideas, the DIM points to articles which discuss
%relatively new ideas more than those which provide specific resources.
These results demonstrate not just that the model is correlated with
citations; it also suggests that the model provides qualitatively
\emph{different} information than citations.

\subsection{Avenues for future work}
The DIM could be made more realistic and more powerful in many ways.
In one variant, individual documents might have their own ``windows''
of influence.  Other improvements may change the way ideas themselves
are represented, e.g. as atomic units, or \emph{memes}
\citep{leskovec:2009}.  Further variants might differently model the
flow of ideas, by modeling topics as birth and death processes, using
latent force models \citep{alvarez:2009}, or by tracking influence
\emph{between} documents, building on the ideas of
\cite{shaparenko:2007} or \cite{dietz:2007}. % citet

We also believe that it would be useful to better understand models like
the DIM in the context of traditional metrics of influence, such as
academic citations, and other metrics of influence, such as usage
data.  Having a better understanding of when this model and
established metrics differ will uncover where our metric may provide
new information that is not yet captured by existing statistics.

\subsection{Next steps}
The work presented in this chapter assumes that the collection of
documents is described by a set of themes, and that these themes
evolve over time.  It describes each document using a mixture over
themes and a vector describing its influence on each of those themes.
This provides a sense of the current of ideas coursing through a
collection of documents.

A limitation of this approach is that it provides too broad a view of
a corpus: it does not provide explicit detail of the underlying story
\emph{within} a collection.  This model describes a corpus as a
collection of topics, and it describes documents as mixtures of themes
and influence weights, but it does not provide any further sense of a
story which changes over time.

In the next chapter we will discuss a model to explore some of these
shortcomings by explicitly modeling the ``story'' within a collection
of text documents.  This approach will use some of the same ideas from
this chapter.  Again we will assume that a collection of text
documents serve as a window into the events within the collection of
historical documents, and again we will encode assumptions by
explicitly modeling them with latent random variables, linked by a
time-series model.  Howver, by modeling the interactions of entities
within the collection explicitly, and applying posterior inference, we
will learn a story about them.

%\section{Acknowledgements}
%The author would like to thank XXXX XXX, XXX XXX, and XXX XXX for
%their helpful comments, feedback, and proofreading.
