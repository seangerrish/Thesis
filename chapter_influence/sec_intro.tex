%\section{}

- Typically in text analysis, documents tell a story.  But this story
is hard to detect without metadata.

- One of the most natural questions we can ask about these documents
-- from the perspective of their historiography -- is which ones are
most influential.

- This happens in scientific research, for example.

One of the most fundamental problems in research is that of finding
which ideas in a field have been the most influential.

This an important and common problem in many fields, including
political science, history, and scientific research.
Influence measurements are used to assess the quality of academic
instruments, such as journals, scientists, and universities; and they
can play a role in decisions surrounding publishing and funding.
Influence measurements are also important for academic research:
finding and reading the influential articles of a field is central to
good research practice.

Modern researchers now have at their disposal a large catalog of
previous work.  Their challenge is finding articles that are important
and interesting while wading through articles that are not.

In this chapter, we will describe an approach to identifying
influential articles in a collection without the use of citations.
The key assumption of our method is that an influential article will
affect how future articles are written and that this effect can be
detected by examining the way corpus statistics change over time.  We
encode this intuition in a time-series model of sequential document
collections.

\subsection*{Measuring influence with citations}

A traditional method of assessing an article's influence is to count
the citations to it. The impact factor of a journal, for example, is
based on aggregate citation counts~\cite{garfield:2002}.  This is
intuitive: if more people have cited an article, then more people have
read it, and it is likely to have had more impact on its field.
Citation counts are used with other types of documents as well.  The
Pagerank algorithm, for example, uses hyperlinks of web-pages to
identify the most influential Webpages on the Internet, and it was
essential to Google's early success in Web search~\cite{brin:1998}.
There is a large literature on these and other methods for citation
analysis and bibliometrics.  See~\cite{osareh:1996} for a review.

Though citation counts can be powerful, they can be hard to use in
practice.  Some collections, such as news stories, blog posts, or
legal documents, contain articles that were influential on others but
lack explicit citations between them.  Other collections, like OCR
scans of historical scientific literature, do contain citations, but
they are difficult to read in reliable electronic form.  Finally,
citation counts only capture one kind of influence.  All citations
from an article are counted equally in an impact factor, when some
articles of a bibliography might have influenced the authors more than
others.

\subsection*{Using text to measure influence}

We will use a text-based approach to measure influence and base our
assumptions on a topic model which allows topics to drift over time in
a corpus~\cite{blei:2006}.

Though our algorithm aims to capture something different from
citation, we will validate the inferred influence measurements by
comparing them to citation counts.  However, we seek a model that is
applicable to collections for which the notion of citation may not
exist.  Therefore, predicting citations is an explicit non-goal.  For
the reader interested in predicting citations, an appropriate solution
might be to propose features and models for predicting future or
heldout citation counts; Tang and Zhang \cite{tang:2009} and Lokker et
al. \cite{lokker:2008} use methods like these; successful features
include the publishing journal's impact factor, previous citations to
last author, key terms, and number of authors
\cite{tang:2009,lokker:2008}.

\subsubsection*{Using text to measure influence}

%interdisciplinary journal such as \textit{Nature} might have influence
%on Neuroscience without having influence on Genomics.  In our model,
%both the topics and influential articles are inferred.
In the next section we describe our model, which is based on the
notion of topics which change over time.  We analyzed one hundred
years of the Proceedings of the National Academy, one hundred years of
\textit{Nature}, and a forty year corpus of articles on computational
linguistics.  With only the language of the articles as input, our
algorithm produces a meaningful measure of each document's influence
in the corpus.

In the rest of this chapter, we will introduce and explore a model to
encode these assumptions.  We begin with a discussion of previous work
aimed at modeling influential documents.  We then describe the
Document Influence Model (DIM), our unsupervised model for determining the
influence of a document using the changes in language used by
documents over time.  We follow this with experiments to both compare
this model with citation counts on three well-known corpora and to
provide the reader with an intuition for the model.