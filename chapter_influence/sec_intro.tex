\label{chapter:influence}

A fundamental problem in research and industry is that of organizing
collections of documents.  In many cases this problem can be reduced
to identifying those documents which have been the most influential.
This is an important and common problem in many fields, including
research in academic fields such as political science, history, and
science.  Influence measurements are used to assess the quality of
academic instruments, such as journals, scientists, and universities;
and they can play a role in decisions surrounding publishing and
funding. These measurements are critical for academic researchers:
find and reading the influential articles of a field is central to
good research practice.  They are also significant in industry, as
regulations such as Sarbanes Oxley require public companies to retain
documents.

E-discovery is another field which faces this challenge of .  As a
recent New York Times article notes the immense need for such tools in
industry:
\begin{quote}
  \emph{[In 1978] the studios examined six million documents at a
    cost of more than \$2.2 million, much of it to pay for a platoon
    of lawyers and paralegals who worked for months at high hourly
    rates.  In January, for example, Blackstone Discovery of Palo
    Alto, Calif., helped analyze 1.5 million documents for less than
    \$100,000...}

  \emph{The economic impact will be huge,” said Tom Mitchell, chairman
    of the machine learning department at Carnegie Mellon University
    in Pittsburgh. 'We’re at the beginning of a 10-year period where
    we’re going to transition from computers that can’t understand
    language to a point where computers can understand quite a bit
    about language} \cite{markoff:2011}.
\end{quote}

The article continues, noting that recent solutions use either
keyword-based search methods or take advantage of metadata such as
citations or links in emails, which can be helpful when available.
Metadata can be a boon for finding the most influential documents in a
collection, but often such metadata is unavailable.

In this chapter, we will describe an approach to identifying
influential articles in a collection without the use of citations.
The key assumption of our method is that an influential article will
affect how future articles are written and that this effect can be
detected by examining the way corpus statistics change over time.  We
will take advantage of the tools discussed in the last chapter by
using them to encode this intuition in a model to measure influence in
sequential collections of documents.

\subsection*{Measuring influence with citations}

A traditional method of assessing an article's influence is to count
the citations to it. The impact factor of a journal, for example, is
based on aggregate citation counts~\cite{garfield:2002}.  This is
intuitive: if more people have cited an article, then more people have
read it, and it is likely to have had more impact on its field.
Citation counts are used with other types of documents as well.  The
Pagerank algorithm, for example, uses hyperlinks of web-pages to
identify the most influential Webpages on the Internet, and it was
essential to Google's early success in Web search~\cite{brin:1998}.
There is a large literature on these and other methods for citation
analysis and bibliometrics.  See~\cite{osareh:1996} for a review.

Though citation counts can be powerful, they can be hard to use in
practice.  Some collections, such as news stories, blog posts, or
legal documents, contain articles that were influential on others but
lack explicit citations between them.  Other collections, like OCR
scans of historical scientific literature, do contain citations, but
they are difficult to read in reliable electronic form.  Finally,
citation counts only capture one kind of influence.  All citations
from an article are counted equally in an impact factor, when some
articles of a bibliography might have influenced the authors more than
others.

\subsection*{Using text to measure influence}

We will use a text-based approach to measure influence and base our
assumptions on a topic model which allows topics to drift over time in
a corpus~\cite{blei:2006}.

Though our algorithm aims to capture something different from
citation, we will validate the inferred influence measurements by
comparing them to citation counts.  However, we seek a model that is
applicable to collections for which the notion of citation may not
exist.  Therefore, predicting citations is an explicit non-goal. 

For the reader interested in predicting citations, an appropriate
solution might be to propose features and models for predicting future
or heldout citation counts; Tang and Zhang \cite{tang:2009} and Lokker
et al. \cite{lokker:2008} use methods like these; successful features
include the publishing journal's impact factor, previous citations to
last author, key terms, and number of authors
\cite{tang:2009,lokker:2008}.  Such research has had measured success:
56\% explained variance \cite{lokker:2008}, and 91.5\% prediction
accuracy \cite{ibanez:2009}.  However, works such as these have
specialized classifiers and restrictive features for narrow
application domains; Lokker et al. \cite{lokker:2008} even note that
their results ``may not be readily transferable to... basic science
articles or journals''. Indeed, Lokker et al.  further noted that
earlier work in their field of predicting citations to medical
journals had only achieved 14\% to 20\% explained variance
\cite{lokker:2008}.

%\subsubsection*{Using text to measure influence}

%interdisciplinary journal such as \textit{Nature} might have influence
%on Neuroscience without having influence on Genomics.  In our model,
%both the topics and influential articles are inferred.

In the rest of this chapter, we will introduce and explore a model to
encode these assumptions.  We begin with a discussion of previous work
aimed at modeling influential documents.  We then describe the
Document Influence Model (DIM), our unsupervised model for determining
the influence of a document using the changes in language used by
documents over time.  We follow this with experiments to both compare
this model with citation counts on three well-known scientific corpora
and a collection of legal opinions to provide the reader with an
intuition for the model. With only the language of the
articles as input, our algorithm produces a meaningful measure of each
document's influence in the corpus.
