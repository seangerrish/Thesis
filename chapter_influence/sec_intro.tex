\section{Introduction}

- Typically in text analysis, documents tell a story.  But this story is hard to detect without metadata.

- One of the most natural questions we can ask about these documents -- from the perspective of their historiography -- is which ones are most influential.

- This happens in scientific research, for example.

Measuring the influence of a scientific article is an important and
challenging problem.  Influence measurements are used to assess the
quality of academic instruments, such as journals, scientists, and
universities, and can play a role in decisions surrounding publishing
and funding.  They are also important for academic research: finding
and reading the influential articles of a field is central to good
research practice.

With unprecedented availability of large collections of scientific
articles, modern researchers have at their disposal a large catalog of
previous work.  Their challenge is to find articles that are important
and interesting.

The traditional method of assessing an article's influence is to count
the citations to it.  The impact factor of a journal, for example, is
based on aggregate citation counts~\cite{garfield:2002}.  This is
intuitive: if more people have cited an article, then more people have
read it, and it is likely to have had more impact on its field.
Citation counts are used with other types of documents as well: the
Pagerank algorithm, which uses hyperlinks of web-pages, has been
essential to Google's early success in Web search~\cite{brin:1998}.

Though citation counts can be powerful, they can be hard to use in
practice.  Some collections, such as news stories, blog posts, or
legal documents, contain articles that were influential on others but
lack explicit citations between them.  Other collections, like OCR
scans of historical scientific literature, do contain citations, but
they are difficult to read in reliable electronic form.  Finally,
citation counts only capture one kind of influence.  All citations
from an article are counted equally in an impact factor, when some
articles of a bibliography might have influenced the authors more than
others.
% Last sentence could be removed

We take a different approach to identifying influential articles in a
collection.  Our idea is that an influential article will affect how
future articles are written and that this effect can be detected by
examining the way corpus statistics change over time.  We encode this
intuition in a time-series model of sequential document collections.

%interdisciplinary journal such as \textit{Nature} might have influence
%on Neuroscience without having influence on Genomics.  In our model,
%both the topics and influential articles are inferred.

We base our model on dynamic topic models, allowing for multiple
threads of influence within a corpus~\cite{blei:2006}. Though our
algorithm aims to capture something different from citation, we
validate the inferred influence measurements by comparing them to
citation counts.  We analyzed one hundred years of the Proceedings of
the National Academy, one hundred years of \textit{Nature}, and a
forty year corpus of articles on computational linguistics.  With only
the language of the articles as input, our algorithm produces a
meaningful measure of each document's influence in the corpus.

In this chapter, we will introduce and explore a model to encode these
assumptions.  We begin with a discussion of previous work aimed at
modeling influential documents.  We then describe the Document
Influence Model, our unsupervised model for determining the influence
of a document using the changes in language used by documents over
time.  We follow this with experiments to both compare this model with
citation counts on three well-known corpora and to provide the reader
with an intuition for the model.
