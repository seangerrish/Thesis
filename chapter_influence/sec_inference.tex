\vspace{-2pt}
\section{Inference and parameter estimation}
\label{section:inference}

Our computational challenge is to compute the posterior distribution
of the latent variables---the sequences of topics and the per-document
influence values---conditioned on an observed corpus.  As for simpler
topic models, this posterior is intractable to compute exactly. We
employ variational methods to approximate it. Variational methods
posit a simpler distribution over the latent variables with free
parameters (called variational parameters).  These parameters are fit
to minimize the KL divergence between the variational distribution and
the true posterior, which is equivalent to maximizing a lower bound on
the marginal probability of the observations. See~\cite{jordan:1999} % citet
for a review of this class of approximate inference methods.

We begin by specifying a variational distribution for the DIM
posterior.  First, the word assignments $z_n$ and topic proportions
$\theta_d$ are governed by multinomial parameters $\phi_d$ and
Dirichlet parameters $\gamma_d$, as in LDA~\cite{blei:2003}.   % citep

The variational distribution for topic trajectories
$\{\beta_{k,1}, \ldots, \beta_{k,T} \}$ is described by a linear
Gaussian chain.  It is governed by parameters $\{ \bv_{k,1}, \ldots,
\bv_{k,T} \}$, which are interpreted as the ``variational
observations'' of the chain.  These induce a sequence of means $\mv_t$
and variances $\vv_t$. \cite{blei:2006} call this a ``variational   % citet
Kalman filter.''

Finally, the variational distribution of the document influence value
$\ell_{d,k}$ is a Gaussian with mean $\ellv_{d,k}$ and fixed variance
$\vlv$.
% ~\footnote{In general, we may abuse notation and refer to the
%  vector $\ellv_{t,k}$ of all document weights at time $t$ for topic
%  $k$; notation should be evident from context.}

% dmb: above, do we fit the variance term or is it fixed?  (in the
% influence variational distribution)
% smg: I've kept the variance term fixed.

% dmb: above, do we really need that footnote?
% smg: if you think it's unnecessary, then probably not.

% dmb: below, do we ever say that "x" is all the latent variables?  do
% we really need this shorthand?
% smg: I can try to 

The variational distribution is
\begin{align*}
  q(\beta, \ell, z, \theta | & \bv, \ellv, \phi, \gamma) = \nonumber \\
  & \hspace{-38pt} \textstyle \prod_{k=1}^K q(\beta_{k,1:T} | \bv_{k,1:T}) \nonumber \\
  & \hspace{-38pt} \textstyle \prod_{t=1}^T \prod_{d=1}^{D_t} q(\theta_{t,d} | \gamma_{t,d})
  q(\ell_{d} | \ellv_{d})
  \textstyle \prod_{n=1}^{N_{t,d}} q(z_{t,d,n} | \vphi_{t,d,n}). \nonumber \\ % \label{eq:variational_posterior} \\
\end{align*}

% dmb: use different notation from d... it looks like the document index
% smg: that's what d is -- which d are you referring to?

% dmb: in general, clean up this equation.  i'm not sure you need the
% first integral.  it suffices to go right to the terms.  and, i think
% that multiple sums will look better than parentheses.
\vspace{-5pt}
Using this variational family, our goal is to maximize the following
lower bound on the model evidence of the observed words $\W$:
\begin{align*}
 \hspace{-0pt} \ln & \textstyle \hspace{3pt} p(\W) \geq \textstyle \sum_T \expectq{\ln p(\beta_{t} | \beta_{t-1})} \nonumber \\
 & \hspace{-5pt} + \textstyle \sum_{T} \textstyle \sum_{D_t} \expectq{\ln p(\ell_{d})} + \expectq{\ln p(\theta_{d} | \alpha)} \nonumber \\
 & \hspace{-5pt} + \textstyle \sum_{T} \textstyle \sum_{D_t} \textstyle \sum_{N_d} \expectq{\ln p( z_n | \theta_d)} \hspace{-1pt} + \hspace{-1pt} \expectq{\ln p(w_n | z_n, \beta_t)} \\
 & \hspace{-5pt} + H(q). \nonumber \\
  \label{eq:elbo}
\end{align*}

\vspace{-20pt}
This bound is optimized by coordinate ascent, with the variational
parameters optimized sequentially in blocks.  These updates are
repeated until the relative increase in the lower bound is below a
threshold.

\paragraph{Topic trajectories.}
The variational update for $\bv$ is similar to that in
\cite{blei:2006}. For each topic, we update the variational Kalman   % citet
``observations'' by applying gradient ascent:
\begin{align*}
\partl{\mathcal{L}}{\bv_{sw}} =
% From beta generative rule:
   & - \textstyle \frac{1}{\sigma^2} \sum_{t=1}^T
     \left( \mv_{tw} - \mv_{t-1,w} - G_{t-1,w} \right) \\
   & \hspace{30pt} \textstyle \times \left( \partl{\mv_{tw}}{\bv_{sw}}
     - \partl{\mv_{t-1,w}}{\bv_{sw}}
     + G_{t-1,w} \partl{\mv_{t-1,w}}{\bv_{sw}} \right) \\
   & + \textstyle \sum_T \left(
       N_{w,t} - N_{t} \zeta_t^{-1}
       \exp( \hat{m}_{\bb_{tw}} + \frac{\tilde{V}_{tw}}{2} ) \right)
       \partl{\mv_{tw}}{\bv_{sw}} \\
   & + \textstyle \frac{1}{\sigma^2} \sum_{t=1}^T
         \partl{\mv_{t-1,w}}{\bv_{sw}}
         \left( H_{t-1,w} - G_{t-1,w}^2 \right) \\
   & + \textstyle \frac{1}{\sigma^2} \sum_{t=0}^{T-1}
         \partl{\mv_{tw}}{\bv_{sw}}
         G_{tw} \vv_{tw}, \\
\end{align*}

\vspace{-29pt}
where
\begin{eqnarray*}
%\mbox{where} \\
\textstyle G_{sn} &=& \expectq{\exp(-\beta_{s,k,n} ) ( \W_{s,k,n} \circ
  z_{s,k,n} ) \ell_{s,k}} \\
\textstyle H_{sn} &=& \mathbb{E}_q \big[ \exp(-2 \beta_{s,k,n}) \left( (\W_{s,k,n} \circ z_{s,k,n} ) \ell_{s,k} \right)^2 \big]. \\
\end{eqnarray*}

\vspace{-14pt}
We expand $H_t$ in the supplementary materials. Note also the
variational parameter $\zeta_t$ and the term
$\partl{\mv_{tn}}{\bv_{sn}}$, both described in \cite{blei:2006}. The  % citet
former can be updated once per iteration with $\zeta_t \gets \sum_w
\exp(\mv_{t,n} + \vv_{t,n}/2)$. The latter can be derived from the
variational Kalman filter updates (see the supplementary materials).

\paragraph{Influence values.}
In the DIM, changes in a topic's mean parameters are governed by a
normal distribution.  As a consequence of this choice, updates for the
influence parameters $\ellv_{t,k}$ solve a linear regression.  In this
regression, documents' words at time $t$ explain the expected topic
drift $\bm{\Delta_{\beta,t,k}} = (\beta_{t+1,k} - \beta_{t,k})$, where
the contributions of each document's words are given by the design
matrix $X = \diag{\exp(-\beta_{t,k})} (\bm W_{t,k} \circ \phi_{t,k}
)$. ($\diag{\vec{x}}$ refers to the matrix having the elements of
$\vec{x}$ on its diagonal.)
%\begin{eqnarray}
%\bm{\Delta_{\beta,t,k,w}} := & \hspace{-34pt} \exp(-\mv_{t,k,w} + \vv_{t,k,w} / 2) \nonumber \\
%& \hspace{-11pt} \times \hspace{2pt} (\mv_{t+1,k,w} - \mv_{t,k,w} + \vv_{t,k,w}). \nonumber
%\end{eqnarray}

The parameter updates for document influence $\ellv_{t,k}$ are
defined, for each time $t$ and each topic $k$, by the
variational normal equation
\[
\ellv_{t, k} \gets \big( \textstyle \frac{\vbv}{\vd} \bm I
  + \expectq{X^T X} \big)^{-1} \expectq{X^T \bm{\Delta_{\beta,t,k}}}.
\]
%\begin{align*}
%\[ \small
%\bm H_{t,k} = \expectq{(\bm W_t \circ z_{t,k})^T \diag{\exp(-2\beta_{t,k})} (\bm W_t \circ z_{t,k})}; \\
%\normalsize
% & \textstyle
% (\bm W_{t,k} \circ \vphi_{t,k})^T \diag{h_{\exp}}
%   (\bm W_{t,k} \circ \vphi_{t,k}) \nonumber \\
% & \hspace{-25pt} \textstyle + \mbox{Diag} \Big( (\bm W_{t,k} \circ \bm W_{t,k} \circ ( \vphi_{t,k} - \vphi_{t,k} \circ \vphi_{t,k}))^T h_{\exp} \Big), \nonumber \\
%\end{align*}
%\]
The expectation $\expectq{X^TX}$ is a matrix with dimension $D_t \times D_t$.  Its elements are
\begin{eqnarray*}
\textstyle {\expectq{X^T X}}_{d,d'} = & \hspace{-15pt} \textstyle \sum_n \exp(-2\mv_{t,k,n} + 2 \vv_{t,k,n}) \\
\textstyle  & \hspace{14pt} \times (w_{t,d,n} w_{t,d',n} \phi_{t,k,d,n } \phi_{t,k,d',n})
\vspace{-20pt}
\end{eqnarray*}
% \vspace{-20pt}
when $d \neq d'$ and
\begin{eqnarray*}
\vspace{-20pt}
\hspace{-10pt} \textstyle {\expectq{X^T X}}_{d,d} = & \hspace{-5pt} \textstyle \sum_n \exp(-2\mv_{t,k,n} + 2 \vv_{t,k,n}) \\
\textstyle  & \hspace{15pt} \times (w_{t,d,n}^2 \phi_{t,k,d,n}) \\
\end{eqnarray*}
otherwise. The expectation $\expectq{X^T \bm{\Delta_{\beta,t,k}}}$
is a $D_t$-dimensional matrix with elements
\begin{eqnarray*}
\textstyle {\expectq{X^T \bm{\Delta_{\beta,t,k}}}}_d = & \hspace{-10pt} \sum_n & \hspace{-10pt} w_{t,d,n} \phi_{t,k,d,n} \nonumber \\
\textstyle && \hspace{-20pt} \times (\mv_{t+1,k,n} - \mv_{t,k,n} + \vv_{t,k,n}/2) \nonumber \\
\textstyle && \hspace{-20pt} \times \exp(-\mv_{t,k,n} + \vv_{t,k,n}/2). \nonumber
\end{eqnarray*}

\paragraph{Topic proportions and topic assignments.}
Updates for the variational Dirichlet on the topic proportions
$\theta_{d,k}$ have a closed-form solution, exactly as in LDA
\cite{blei:2003}; we omit details here.

The variational parameter for each word $w_n$'s hidden topic $z_n$ is
the multinomial $\phi_n$.  We solve for $\phi_{n,k}$ by the
closed-form updates
\begin{eqnarray*}
%\log(\phi_{w,k}) & \gets & \Psi(\gamma_k) + \mv_{t, k,w} \nonumber \\
%& & \hspace{-60pt} + \frac{1}{\sigma^2} \bm w_{t} \ellv_{d_w,k} \big( \exp(-\mv_{t,k,w} + \vv_{t,k,w} / 2) \nonumber \\
%    & & \hspace{-10pt} \times  (\mv_{t + 1,k,w} - \mv_{t,k,w} + \vv_{t,k,w}) \big) \nonumber \\
%   & & \hspace{-60pt} - \frac{1}{\sigma^2} \bm w_{t} \exp(-2 \mv_{t} + 2 \vv_t) \big( 
%        \ellv_{d} \times (\bm W_t \circ \phi_{t,k}^{\mbox{L}} \ellv_{t,k})_w \nonumber \\
%   & & \hspace{55pt} + \bm w_{t} \phi_{w,k}^{\mbox{L}} \ellv_{d}^2 - \ellv_{d}^2 - \sigma_l^2 \big) \nonumber \\
\small
\textstyle \log(\phi_{n,k}) & \gets & \Psi(\gamma_k) + \mv_{t,k,n} \nonumber \\
& & \hspace{-68pt} + \frac{1}{\sigma^2} w_{t} \ellv_{d_n,k} \exp(-\mv_{t,k} \hspace{-1pt} + \hspace{-1pt} \vv_{t,k} / 2) (\mv_{t+1,k} \hspace{-1pt} - \hspace{-1pt} \mv_{t,k} \hspace{-1pt} + \hspace{-1pt} \vv_{t,k})  \\
   & & \hspace{-68pt} - \frac{1}{\sigma^2} w_{t,n} \Big[ \ellv_{d_n,k} \exp(-2\mv_{t,k} + 2 \vv_{t,k}) \\
& & \hspace{-22pt} \times (\bm W_{t,n,\setminus{d_n}} \circ \phi_{t,n,k,\setminus{d_n}}) \ellv_{t,k,\setminus d_n} \Big] \\
& & \hspace{-68pt} - \frac{1}{\sigma^2} w_{t,n}^2 \exp(-2\mv_{t,k} + 2 \vv_{t,k}) (\ellv_{d,n,k}^2 + \sigma_l^2) \\
%\label{eq:phi_update}
\end{eqnarray*}

\vspace{-20pt}
where $\Psi$ is the digamma function. Solving the constrained
optimization problem, this update is followed by normalization,
$\phi_{w, k} \gets \frac{\phi_{w, k}}{\sum_K \phi_{n, k}}$.
