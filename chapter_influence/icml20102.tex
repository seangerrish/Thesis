\documentclass{article}
\usepackage{graphicx}
\usepackage{amsmath}
\usepackage{amsfonts}
\usepackage{amssymb}
\usepackage{color}
\usepackage{subfigure}
\usepackage{gensymb}
\usepackage{microtype}
\usepackage{times}
\usepackage{mdwlist}
\usepackage[small,compact]{titlesec}
\usepackage{natbib}
\usepackage{algorithm}
\usepackage{algorithmic}
\usepackage{hyperref}
\newcommand{\theHalgorithm}{\arabic{algorithm}}
\usepackage[accepted]{icml2010} 
\usepackage{times}




\long\def\comment#1{}

\newcommand{\fix}{\marginpar{FIX}}
\newcommand{\new}{\marginpar{NEW}}
\DeclareMathOperator*{\GP}{GP}
\DeclareMathOperator*{\GaP}{GaP}
\DeclareMathOperator*{\DP}{DP}
\DeclareMathOperator*{\GEM}{GEM}
\DeclareMathOperator*{\Bet}{Beta}
\DeclareMathOperator*{\IBP}{IBP}
\DeclareMathOperator*{\BP}{BP}
\DeclareMathOperator*{\BeP}{BeP}
\DeclareMathOperator*{\Bern}{Bernoulli}
\DeclareMathOperator*{\Disc}{Discrete}
\DeclareMathOperator*{\Mult}{Multinomial}
\DeclareMathOperator*{\Pois}{Poisson}
\DeclareMathOperator*{\Gam}{Gamma}
\DeclareMathOperator*{\Dir}{Dirichlet}
\DeclareMathOperator*{\PG}{NB}
%\let\oldthebibliography=\thebibliography
%\let\endoldthebibliography=\endthebibliography
%\renewenvironment{thebibliography}[1]{
%\begin{oldthebibliography}{#1}
%\setlength{\parskip}{0ex}
%\setlength{\itemsep}{0ex}
%}
%{
%\end{oldthebibliography}
%}
\icmltitlerunning{Indian Buffet Process Compound Dirichlet Process}

\begin{document}
\twocolumn[
\icmltitle{The IBP Compound Dirichlet Process \\ and its Application to Focused Topic Modeling}

% It is OKAY to include author information, even for blind
% submissions: the style file will automatically remove it for you
% unless you've provided the [accepted] option to the icml2010
% package.
\icmlauthor{Sinead Williamson}{saw56@cam.ac.uk}
\icmladdress{Department of Engineering, University of Cambridge,
            Trumpington Street, Cambridge, UK}
\icmlauthor{Chong Wang}{chongw@cs.princeton.edu}
\icmladdress{Department of Computer Science, Princeton University,
            35 Olden Street, Princeton, NJ 08540, USA}
\icmlauthor{Katherine A. Heller}{heller@gatsby.ucl.ac.uk}
\icmladdress{Department of Engineering, University of Cambridge,
            Trumpington Street, Cambridge, UK}
\icmlauthor{David M. Blei}{blei@cs.princeton.edu}
\icmladdress{Department of Computer Science, Princeton University,
            35 Olden Street, Princeton, NJ 08540, USA}
% You may provide any keywords that you 
% find helpful for describing your paper; these are used to populate 
% the "keywords" metadata in the PDF but will not be shown in the document
\icmlkeywords{topic modeling, IBP}

\vskip 0.3in
]

% The \author macro works with any number of authors. There are two commands
% used to separate the names and addresses of multiple authors: \And and \AND.
%
% Using \And between authors leaves it to \LaTeX{} to determine where to break
% the lines. Using \AND forces a linebreak at that point. So, if \LaTeX{}
% puts 3 of 4 authors names on the first line, and the last on the second
% line, try using \AND instead of \And before the third author name.



\begin{abstract}
Identifying the most influential documents in a corpus is an important
problem in many fields, from information science and
historiography to text summarization and news aggregation.
Unfortunately, traditional bibliometrics such as citations are often
not available.  We propose using changes in the thematic content of
documents over time to measure the importance of individual documents
within the collection.  We describe a dynamic topic model for both
quantifying and qualifying the impact of these documents and validate
this model empirically.
\end{abstract}

\end{document}
