\chapter{Introduction}

Information is all around us.  Society interacts with this information
in a complicated dance: information affects what we do, and we create
further information -- books, scientific papers, legislation, and
tweets -- as a result.  Information influences everything we say,
think, and do.

The goal of this thesis is to describe several new statistical models
for better understanding the interplay between text and society.

\section{Social science data analysis}

\subsection*{The availability of observational social science data on a massive scale}
  - observational social science data is available on a massive scale. \cite{lazer:2009}
  - 

\subsection*{Text}
Text data is the low-hanging fruit of most social science research
questions.  Its ubiquity is due to the ease with which it
can---must---be created, digitized, replicated, and stored.  The rate
is staggering.  In 2008, the Internet was growing at a rate of several
billion webpages per day \cite{googleblog:2008}.

Because of the ubiquity of text data, it provides most practitioners
with an extremely rich source of data: documents, one of the basic
units of information in text analysis, are an observation in an
extremely high-dimensional and interpretable space
\cite{changrtl:2009}.

- one of the central themes of this thesis is that text data serve as an observation of an underlying story underlying decisions and politics.

- often

\subsection*{Time-series data}

\subsection*{Dyadic data: networks and interaction between entities}

\subsection*{Dyadic data: networks and interaction between entities}

\subsection*{The insufficiency of traditional methods}

\subsection*{Causal inference in modern observational setting}

\section*{The role of statistical machine learning}

\subsection*{Probabilistic latent variable models and graphical models}
\cite{pearl:1985}

\subsection*{Tools and abstractions for probabilistic inference}

\subsection*{Ability to perform inference on complicated datasets}

\section{Organization}

By the end of this thesis, the reader should have a better
understanding of several models available to social scientists.
Perhaps more importantly, the reader will be prepared to design his or
her own latent-variable model for similar applications. To this end,
we will provide a lower level of detail about latent-variable models
in the early chapters of this thesis when it is appropriate to help
the reader understand the material.

We begin by outlining preliminary material in Chapter 2, outlining the
statistical ``primitives'' that we use as building blocks in later
chapters, including tools for text analysis and time-series analysis.

In chapter 3 we introduce a model for discovering important and
influential documents in a time-series collection.

In Chapter 4 we discuss legislative voting, a domain in which text
additional data is present but the relationship between documents and
lawmakers enables us to blah blah.

In chapter 5 we will use the ideas developed in chapters 3 and 4 to
discover a more sophisticated latent story among documents, by
developing a model of the relationships between countries over time.

We discuss details of a new variational inference algorithm in
Appendix A, which can be treated as a stand-alone contribution of this
thesis.  Further supplementary information for this thesis is provided
in Appendix B.
