\chapter{Quantitative methods for social research in the digital age}

Quantitative social scientists often attempt to understand the
behavior of political entities.  The digital age has brought to these
researchers a deluge of digital records---particularly in the form of
text---.  This avalanche of data provides more information to these
scientists than they have had in the history of mankind.  Researchers
are now able to pore over digital copies of all legally binding
opinions written by United States Supreme Court Justices, or the text
of thousands of bills voted on by members of Congress.  Even these
numbers are dwarfed by the hundreds of thousands of newspaper articles
written about the very bills written about international relations.
Unfortunately, this flood of information obscures the very insights
these researchers aim to discover.  Researchers trying to make sense
of these collections are subject to the high costs of time spent
poring over these collections in search of the few key insights.

 % example questions: how to find phenomena in collections of text
 % how to 
% The goal of this thesis is to demonstrate that complex, meaningful patterns of
% information can be gleaned
% Society interacts with this information in
% a complicated dance: information affects what we do, and we create
% further information -- books, scientific papers, legislation, and
% tweets -- as a result.  Information influences everything we say,
% think, and do.
The goal of this thesis is to describe several new statistical models
to illustrate this -- and at the same time to provide tools for data
consumers, practitioners, and researchers\footnote{Throughout this
  work, I will refer to (data) consumers, practitioners, and
  researchers.  Consumers are those who use processed results from
  data analysis in some form but do not interact directly with the
  data itself (for example, consumers may be directors in governments
  or companies who make decisions based on these data).  Practitioners
  are those who apply existing methods for data analysis, possibly
  tweaking or combining these methods to answer specific questions
  (such as database engineers or lab assistants). Researchers are
  those who research entirely new methods for data analysis.  Note
  that a social scientist may be a researcher in his or her field but
  a practitioner in the field of data analysis.} to better
understanding society through collections of text documents.  I will
focus on three high-level research questions that dovetail off one
another to illustrate the flexibility and interpretability of latent
variable models in large-scale settings.

A central premise of this thesis is that these complex but
interpretable patterns exist in nearly every collection of text
documents, and that these patterns be discovered automatically to
describe decisions and behavior of actors in the collection.  I will
ground this discussion with the development of several specific models
but take the position throughout this thesis that these methods draw
from a suite of common tools which can be used again to construct
models to address alternative questions.  I will use one of these
projects as a running example through this chapter.

\section*{Tools for analysis in social science}

Political scientists and historians study world history with a
magnifying glass.  Existing methods require painstaking study of
individual documents.   make it challenging for them to I now discuss one of these
models and the problem it solves to be concrete through the rest of

In a later chapter I will outline the method for inferring these relationships.

the introduction.

It also serves as a ``model of central
tendancy'', in that it captures the intuition behind the other models
we will discuss.


But first, we turn to some of the challenges in analyzing collections like this.

\subsection*{Digital text collections and the deluge of information}
% The availability of observational social science data on a massive scale
Observational social science data is available on a massive scale
\cite{lazer:2009}. The National Archives, which collects information
from over 500 federal agencies, is digitizing its collection of twelve
\emph{billion} documents
\cite{national_archives:2012a,national_archives:2012b}.

This thesis will focus largely on text data.  Text data is the
low-hanging fruit of most social science research questions.  It is
ubiquitous because it can---indeed, it must---be easily created,
digitized, and stored.  Just as text data is invaluable, the rate of
growth of these text collections is staggering.  A single newspaper
like the \emph{New York Times} publishes hundreds of thousands of
articles each decade.  Of the National Archive's collection, billions
of its documents are text
\cite{national_archives:2012a,national_archives:2012b}.  The rate of
growth for the World Wide Web is even more staggering.  As far back as
2008, the Internet was already growing at a rate of several billion
webpages per \emph{day} \cite{googleblog:2008}.

%Because of this ubiquity and
%richness of text data, it is an invaluable resource to researchers.

\subsection*{The Curse of Dimensionality and the Blessing of Big Data}

%  One of the most interesting 
% - information in digital age -- this is the first time we have had so much data.  This flood is unmanageable in two senses:
%   - first, data is high-p.  We can solve this by using statistical methods to summarize these observations.  
%   - second, data is large-N.  We can solve this with computers.  However, note that only when N is large enough can we say anything useful about consistent patterns.
One of the primary goals

%   - one of the primary goals of the social scientist is to distill
%   meaningful patterns from these collections of data.  With this flood
%   of information, this can only be done in two ways:

%   - by sampling this information -- in which case he may choose exemplary cases.
%   - by distilling consistent patterns in the data which are useful.
%   - A quantitative treatment will allow us to precisely describe these patterns.  The interpretation of these results is handled by the researcher.
%     - Along these lines, this thesis will take


\subsection*{Time-series data}
One of the most basic questions in any collection of text documents
is, ``which documents are the most important or influential''?

In this research we do not address the question of causality.  Causal
analysis is a rich field with

\subsection*{Dyadic data: networks and interaction between entities}
This question

\subsection*{Probabilistic latent variable models and graphical models}

\subsection*{The role of statistical machine learning}

The basic unit of text analysis---a document---is an observation in an extremely
high-dimensional space.%  \cite{changrtl:2009}.
One of the central themes of this thesis is that text data serve as
an observation of an underlying story underlying decisions and
politics.

- a computer-based treatment is \emph{necessary} if we want to handle
large collections of data.  - a quantitative treatment is convenient
because it is the language of and because it offers an explicit way to
formalize our assumptions.

Traditional methods cannot work because there are simply too many documents.  This means that researchers necessarily ignore huge swaths of data.

% - A need to  perform inference on complicated datasets

We will address these in the  next chapter.
\cite{pearl:1985}

% - Tools and abstractions for probabilistic inference

\section{Organization}

By the end of this thesis, the reader should have a better
understanding of several models available to social scientists.
Perhaps more importantly, the reader will be prepared to design his or
her own latent-variable model for similar applications. To this end,
we will provide a lower level of detail about latent-variable models
in the early chapters of this thesis when it is appropriate to help
the reader understand the material.

I provide preliminary material in Chapter 2, outlining the statistical
``primitives'' that I will use as building blocks in later chapters.
These primitives include statistical tools for working with text data,
time-series data, and dyadic data.

\paragraph{Identifying influential documents.} In Chapter 3 I
introduce a model for discovering important and influential documents
in a collection of documents which has grown over time.  In
collections of legal opinions, one of the most common questions posed
by jurists\footnote{In political science, the word ``jurist'' often
  refers to a judge, as opposed to a member of the public who votes on
  the outcome of a trial.} is, ``which legal opinions have set the
most precedent?''  This question plays a central role among
researchers and archivists in fields outside of law and even motivated
the algorithm behind Larry Page and Sergey Brin's PageRank algorithm,
which recursively measures the influence of Webpages, as measured by
the hyperlinks between Webpages
\cite{garfield:1992,brin:1998,garfield:2002}.  In many real-world
scenarios, however, explicit citations or hyperlinks are missing, and
researchers only see the most basic metadata: documents' timestamps.
I will validate this model on a set of several datasets, including several collections
of academic articles and a set of opinions written by judges in the
New York Appellate Courts system.

\paragraph{Inferring history from a collection of newspaper articles.}
In Chapter 4 I outline a model to address the question posed above, to
better understand the relationships between countries over time and
fit this to a collection of New York Times articles.  I demonstrate
that this method discovers a more sophisticated latent story among
documents, by developing a model of the relationships between
countries over time.  As with the method in Chapter 3, this collection
has only the text of these articles, which I augment with external
information.  In this chapter I also incorporate important ideas from
the field of dyadic spatial models, which can play a role in modeling
various social science phenomena.

% I will describe a method to discover
% which documents are influential on the development of a collection of
% text documents \emph{without} the use of metadata such as
% citations---where by an ``influential'' document I mean one that uses
% language which is adopted by others in the collection.

\paragraph{Inferring lawmakers' preferences.}
In some cases, documents are labeled with some indication of
lawmakers' preferences.  This happens when jurists vote in favor
of---or against---opinions In the United States Congress, Senators and
Representatives publicly state their opinions about bills passing
through Congress.  While traditional methods provide some sense of
this I will describe a method which uses the text of each bill to
predict how lawmakers will vote on information attached to text
documents, such as information about peoples' opinions of them?  I
After this


In Chapter 5 I will extend existing spatial models to better
understand how Congressmen feel about different issues.  I begin with a
model which is able to predict how lawmakers will vote on previously-unseen
bills. I also describe a method for learning lawmakers' issue preferences 
from their votes on different issues.

I discuss details of a new variational inference algorithm -- which
was used in Chpter 5--in Appendix A; this Appendix can be treated as a
stand-alone contribution of this thesis.  I provide additional
supplementary information for this thesis in Appendix B.
