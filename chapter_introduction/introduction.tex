\chapter{Introduction}


Information is all around us.  Society interacts with this information
in a complicated dance: information affects what we do, and we create
further information -- books, scientific papers, legislation, and
tweets -- as a result.  Information influences everything we say,
think, and do.

The point of this thesis is not to prove this point; that information
influences us in everything we do is self-evident.  This thesis will
describe a number of quantitative methods for modeling the influence of text on society.

We begin by ...

We also ...

\section{Influence and decision making}

\section{The availability of observational social science data on a massive scale}

\subsection{Text}
  Text data is the low-hanging fruit of most social science research
  questions.  Its ubiquity is due to the ease with which it can be
  created, digitized, replicated, and stored.  At the same time, it
  provides most practitioners with an extremely rich source of data:
  documents, one of the basic units of information in text analysis,
  are an observation in an extremely high-dimensional and
  interpretable space \cite{changrtl:2009}.

  The Internet is growing at a rate
  of several billion pages per day \cite{googleblog:2008}.

\subsection{Dyadic data: networks and interaction between entities}

\subsection{Time-series data}

\subsection{The insufficiency of traditional methods}

\subsection{Causal inference in modern observational setting}

\section{The role of statistical machine learning}

\subsection{Probabilistic latent variable models}

\subsection{Tools and abstractions for probabilistic inference}

\subsection{Ability to perform inference on large-scale datasets}

