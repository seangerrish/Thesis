In this section I outline the various sources of help I have received
in completing this thesis.

\subsection*{Funding}
This work was supported by grants secured by my advisor from the
Office of Naval Research, ONR 175-6343, NSF CAREER 0745520, AFOSR
09NL202, the Alfred P. Sloan foundation, and a grant from Google.  I
was also supported by a fellowship provided by Princeton to first-year
graduate students.  I was supported by the Princeton Computer Science
department's teaching assignments during my second year.

Various outside organizations have provided funding for travel to
conferences for presenting this work.  This includes scholarships
provided by ICML, NIPS, and the Machine Learning Summer School
organized at Cambridge in 2009, all of which have supported me with
funding for transportation, lodging, and/or registration.  Google,
Inc. provided funding for travel to NIPS 2010; and Facebook,
Inc. provided funding for travel to ICML 2010.  Princeton University's
Dean's Fund for Scholarly Travel and Princeton's School of Engineering
and Applied Sciences Graduate Travel Funds both provided funds for
travel to NIPS 2011.

\subsection*{School}
Foremost, I owe my advisor, David Blei, many thanks for his mentorship
and support for the past four years.  The preponderance of this
mentorship has been on research, but Dave's support during the
formidable ``middle years'' also helped me to press through the
program.  One of the things that stands out with Dave's mentorship is
his steadfast insistence that we continue on a project that has been
started.  This was a healthy counterweight to my ``fail fast''
philosophy, which, retrospectively, may have led to a very
unproductive research career (Dave was also able to differentiate
between disillusionment and failure, which helped in this process).

This was exemplified by a diagram Dave pointed to on his board one
day.  ``This is the process of research.  This graph plots excitement
with your research project as a function of time,'' he said, pointing
to the curve.  ``You start out excited about your project, but after a
while you get sick and tired of it.''  He pointed to the precipitous
dip in the curve.  ``You'll end up sick of your project at this time
-- it happens to everybody -- but if you keep on pushing forward, then
you'll end up with good research.''  The curve moved back upward, past
the dip.  This diagram was helpful.  It was fairly obvious, and I knew
this curve well.  But explicitly seeing that other researchers go
through the same thing was helpful nonetheless.  (It also helped
knowing that Dave did not draw this diagram expressly for me; it was
on his board from an earlier meeting he'd had.)

I would also like to thank the various researchers in the field of
machine learning and the social sciences who have served as mentors or
inspirations, whether they knew it or not.  One of these was Leon
Bottou, from whom I learned more than I should admit while TAing for
him. Leon's deep understanding of mathematics, pleasant manner, and
modesty are testaments to his character.  I also want to thank Kevin
Quinn and Mark Gergen for hosting me for a month at Berkeley to fit
several models over the New York Appellate Courts.  Finally, my time
at JSTOR hosted by Clare Llewellyn, John Burns, Michael Krot, and Ron
Snyder was a valuable experience.

I owe many thanks to a number of current and former graduate students
in my research lab and our broader research group, who have helped me
in this program in various ways.  They have served as sounding boards
and have served as excellent role models.  The graduate students
include Jordan Boyd-Graber Ying, Jonathan Chang, Chong Wang, Indraneel
Mukherjee, Gungor Polatkan, Lauren Hannah, Matthew Hoffman, Sam
Gershman, Melissa Carroll, Berk Kapicioglu, Prem Gopalan, Allison
Chaney, and Rajesh Ranganath.  The post-doctoral researchers have been
equally as helpful; they include David Mimno, John Paisley, and Jeremy
Manning.

% Our post-doctoral researchers were equally as helpful.  These
% post-docs have included David Mimno, who is an exemplar of how to
% succeed in Academia after graduate school; Jeremy Manning, who ; and
% John Paisley, who somehow always seemed to be interested in working on
% almost exactly the same application as I, and who could figure out how
% to perform inference on anything, anywhere.

%  These have included Jordan Boyd-Graber, who set a great my first year and kept tips on
% organization .  Jonathan Chang's stupendous music collection kept me
% entertained while I was writing my thesis, and his research stood as a
% recurring model to me of the type of projects I could (and would) work
% on. Chong Wang was extremely helpful throughout my entire time
% here. Chong served as model graduate student, putting out stellar
% research that I can only hope to reproduce one day.  I enjoyed many
% inspirational chats with Indraneel Mukherjee about mathematics,
% startups, finance, and our lives beyond graduate school.  I shared
% many similar chats with Gungor, who was in many ways a kindred spirit,
% sharing the same interests in applied work.  Sam Gershman ....  Lauren
% Hannah, Matthew Hoffman, wh, I am extremely grateful to have worked with
% Allison Chaney, whose reminded me just how; and Rajesh Ranganath, who
% has already demonstrated a solid grasp of the field.

\subsection*{Committee}
I would like to thank my committee, who spent the time to provide me
with helpful feedback for this thesis.  Their comments were extremely valuable.
\begin{itemize}
  \item David Blei (reader / advisor)
  \item Rob Schapire (reader)
  \item Hanna Wallach (reader)
  \item Matthew Salganik (non-reader)
  \item Kosuke Imai (non-reader)
\end{itemize}

\subsection*{Friends and Family}

There are various outside mentors in my life whom I consulted during
the ``middle years'' (circa 2010) to re-orient myself in the graduate
program. These included Arash Baratloo, who went through similar
periods in graduate school; Jeffrey Oldham, who pointed out the
intangible benefits in a PhD; and Doug Beeferman, who provided an
honest and fair point of view from the ``other'' side.  I also want to
thank Ricky Wong, who reminded me that the work I am doing is
interesting to people outside of Academia (by actually using these
tools at his startup) and for providing topics from a biology textbook
that I reference in Chapter~\ref{chapter:introductory_material}.

I also want to thank those who wrote letters of support for my
applications to grad school.  These included Dan Rubinstein; who
provides incredibly direct, honest, and sound career advice; and who
is an excellent role model; Dragomir Radev, whose work and mentorship
in computational linguistics and natural language processing pointed
my career in its current direction; and Mark Skandera, who mentored me
in math reserch as an undergraduate and continues to be a good friend.

My fiancee Sarah was immensely helpful in keeping me sufficiently
distracted during this period.  Her encouragement and patience was
invaluable in keeping me happy during the last few years.

My parents deserve many thanks for their support in the past three
decades.  Those early years were particularly important, and they set
the stage for my interest in science, engineering, and math---let
alone stressing the importance of school.

I may have never become interested in computer science if my older
brother Josh hadn't shown me things like division, Robot Odyssey and
Rocky's Boots, BASIC, fractals, multi-user dungeons, bulletin boards,
and countless other geeky things (okay, maybe I would have learned
about division in third grade, but it's much cooler if your older
brother shows you).  Growing up in the wake of Josh was invaluable.
His books and articles on logic puzzles, algebra, mathematical proofs,
and 3D computer graphics gave me a chance to learn some of these ideas
on my own time.  Josh also encouraged me to push ahead with my PhD
during the times that it would have been easiest to take another path.

My brother Jason has provided me with with many experiences that
transcended math and science.  Jason's mischief in the first year of
my program was probably a good thing for me, as it made me aware
that there are important things outside of research that are worth
thinking about; this provided a buoyant relief from research.  Jason's
steadfast encouragement was also important in helping me to continue
my PhD.

Finally, I want to thank my little sister Kim.  Kim, like Jason,
provided a buoyant relief from work.  She talked me through my first
year of graduate school and inspired me through the remaining years.
